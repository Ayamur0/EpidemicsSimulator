\section{Achieved Results}
In chapter \ref{cha:general_principles} the theory of epidemics was introduced. The two main parts of epidemics were discussed, networks and diseases. Three existing models and their respective limitations were explained, a simple tree-based model, the SIR Model and the SIS Model. These were  combined and extended to create the custom model used in this work. This custom model was explained in detail in section \ref{sec:custom_model}.

An app was created that uses the custom model to allow for simulations of epidemics. The app includes functions for modeling social networks, creating diseases, running simulations with the created networks and diseases and collecting statistics about the simulation. The general functionality of the app has been outlined in chapter \ref{cha:implementation} and its implementation is described in chapters \ref{cha:network_generation}, \ref{cha:network_display} and \ref{cha:visual_simulation}, which explain the process and algorithms involved in creating the app.

After creating the app some experiments were conducted to support the theories discussed in chapter \ref{cha:general_principles}. These experiments were performed using the developed app. The results of the simulations coincided with the theories of how the diseases are expected to behave. This indicated the importance of factors like the reproductive number $R_0$ in epidemics. The findings of these experiments and theories can be used to better prepare for real-world scenarios of epidemics. They help to understand the behavior of diseases and their lifetime in complex social networks.

\section{Outlook}
\subsection{Extensibility of the Results}
The network editor can be extended to allow precise control over which nodes are connected to which. This requires a solution that still allows fast creation of large networks while providing such fine control over the connections.

For very large networks, the computation can take a long time. Tests showed that networks with \textasciitilde  100,000 nodes took almost 30 minutes to generate. A big factor for this long time is that Python is an interpreted language, which makes such calculations relatively slow compared to the same calculations in a compiled language like C++. The app could be reimplemented in C++ which would reduce the computing time, but since C++ does not provide a package like plotly, a custom implementation for displaying the graphs is necessary, which can be time consuming to implement.

\subsection{Transferability of the Results}
The results achieved using the simulation can be applied to real-world scenarios. These simulations can help to predict how an epidemic will evolve over time. This can be used to test how certain factors affect the spreading of the disease to determine what countermeasures are necessary to suppress the disease and ensure the safety of the population. One example might be reducing the number of connections in the network, which could be achieved in the real world by implementing social distancing measures. Another factor that can be tested is whether a vaccine that reduces the infection probability by a certain amount is sufficient to make the disease die out or what percentage of people would need to be vaccinated to achieve this.

Without tools to simulate the behavior of diseases, it is almost impossible to predict what will happen, making it difficult to decide on the necessary countermeasure until it might be too late. This is why tools for simulation are necessary in the field of epidemics.
