Algorithmic Game Theory is a rapidly growing field. In this paper the basic principles such as different types of games, solution strategies, Nash equilibria and computational problems were introduced. Today there are still many research topics and developments in this area. Things like whether or not PPAD-complete problems are easier than NP-complete problems are still not definitely proven.

Other current research areas within Algorithmic Game Theory include the design of algorithmic mechanisms that are robust to strategic manipulation and on the computational complexity of these mechanism design problems, a topic also covered by Nisan et al. \autocite{agt}.

Another area of research is dynamic games, which are games that evolve over time. These games can be used to model concepts such as auctions or negotiations. Negotiations, for example, have applications in many areas, from social areas to cloud computing. The current research focuses on developing algorithms to analyze these games, as well as understanding the impact of time and strategic behavior in these games.

Machine Learning is another area of growing importance and is closely related to Algorithmic Game Theory. Current research focuses on the intersection of these two fields and the development of algorithms for learning games that can be used to analyze the behavior of machine learning algorithms. The use of Machine Learning to improve game theoretic analysis is another area of current research. Dey \autocite{MLAndAGT} shows an example use case for combining Machine Learning and Algorithmic Game Theory and Hazra and Anjaria \autocite{agtInDL} discuss applications of Algorithmic Game Theory in the field of Deep Learning.

Overall Algorithmic Game Theory includes many areas that still have interesting topics that are currently being researched. The field is constantly evolving and only growing in importance, making it a very interesting area with many topics still available for research.