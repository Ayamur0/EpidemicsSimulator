\section{Motivation}
An epidemic outbreak can have a major impact on the world. Past epidemics have shown that
if we do not know how to counteract an epidemic outbreak and are not prepared for 
such cases, a new disease can cause the death of major parts of the world. The black
death caused the death of about 30\% to 60\% of all Europeans (75-200 million)
during the 1300s \cite{blackDeath}.
To better understand such scenarios simulations play an important role since studying real
cases of epidemic outbreaks is difficult as there are only so many in the history of humans.
Also it is important in case an outbreak happens to be able to simulate the next few days/weeks
to accurately predict how the epidemic will evolve.

For this reason this work will discuss an approach to simulate such epidemics by modeling
networks of persons and then simulating the spreading of diseases with different properties
in these networks.

\section{Problem definition}
A model of the social network the disease is spreading in is curcial to the simulation.
Depending on the transmission method of the disease this network can be highly connected in 
the case of a disease with airborne transmission or have only few connections for diseases
that are for example sexually transmitted. In addition to that different diseases can spread
completely different in the same social network even if the have the same transmission method
because the characteristics of the disease also play an important role in how it spreads. 
As suggested by Easley and Kleinberg \cite{networks}
the transmission of computer viruses works in a similar way an thus is also able to be modeled
by using networks.

The networks that can be created using the app must be able to model different social
networks. The important part for the epidemics simulation is the modeling of the amount
of contacts with other humans each human has. Since each group can contain a significant amount
of members an efficient method for creation networks with large amounts of nodes is required, which
still allows to model most social networks.

A method to visually display these networks in a clear way is required. The visual 
representation must still be usable with large amounts of nodes (eg. over 100,000 nodes).
To make the visualization of the network more usable some settings need to be provided
to alter the displayed network, like hiding certain connections or nodes. It also needs
ways to represent the current state of the network in respect to the spread of the diseases.

The app also needs to allow for creation of multiple diseases with different properties.
To simulate various epidemic scenarios properties like the infectiousness, duration of illnes
or fatality need to be editable. 

The app should be able to simulate multiple diseases at the same time. The simulation needs
to take into account which humans have contact with each other and then simulate the spreading
of the diseases according to the properties of each disease and group of humans.

During the simulation the app will collect statistics to allow a review of key information
after a simulation, like the amount of new infections over time.