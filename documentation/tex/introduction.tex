\section{Motivation}
An epidemic outbreak can have a major impact on the world. Past epidemics have shown that
if we do not know how to respond to an epidemic outbreak and are not prepared for 
such cases, a new disease can wipe out large parts of the world. The black
death caused the death of about 30\% to 60\% of all Europeans (75-200 million)
during the 1300s \cite{blackDeath}.
To better understand such scenarios, simulations play an important role, as the study of real
cases of epidemic outbreaks is difficult since there are only so many in the history of humans.
Also, in the event of an outbreak, it is important to be able to simulate the next few days/weeks
in order to accurately predict how the epidemic will evolve.

For this reason this work will discuss an approach to simulate such epidemics by modeling
networks of people and then simulating the spreading of diseases with different characteristics
in these networks.

\section{Problem definition}
A model of the social network in which the disease is spreading in is crucial to the simulation.
Depending on the transmission method of the disease, this network may be highly connected in 
the case of a disease with airborne transmission or have only few connections for diseases
that are sexually transmitted. In addition, different diseases can spread
completely different in the same social network even if they have the same transmission method
because the characteristics of the disease also play an important role in how it spreads. 
As suggested by Easley and Kleinberg \cite{networks}
the transmission of computer viruses works in a similar way can therefore also be modeled
using networks.

The networks that can be created using the app must be able to model different social
networks. The most important part for the epidemics simulation is the modeling of the amount
of contacts with other people each person has. Since each group can have a significant number
of members an efficient method for creating networks with large amounts of nodes is required that
still allows to model most social networks.

A method to visualize these networks in a clear way is needed. The visual 
representation must still be usable with large amounts of nodes (e.g. over 100,000 nodes).
To make the visualization of the network more usable, some settings must be provided
to modify the displayed network, such as hiding certain connections or nodes. It also needs
ways to represent the current state of the network in respect to the spread of the diseases.

The app also needs to allow for creation of multiple diseases with different characteristics.
To simulate various epidemic scenarios properties like the infectiousness, duration of illness
or fatality need to be editable. 

The app should be able to simulate multiple diseases at the same time. The simulation needs
to take into account which humans have contact with each other and then simulate the spreading
of the diseases according to the characteristics of each disease and group of people.

During the simulation the app will collect statistics that allow a review of key information
after the simulation, such as the number of new infections over time.