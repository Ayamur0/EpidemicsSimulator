This paper aims to serve as an introduction to Algorithmic Game Theory. Algorithmic Game Theory is an area at the intersection of Game Theory and Computer Science. Algorithmic Game Theory analyzes and designs algorithms for strategic interactions between multiple agents who each want to maximize their payoffs and calculate their best strategies or best responses. These strategic interactions can be modeled as algorithmic games.

The main goal of Algorithmic Game Theory is to find algorithmic solutions to these often complex games. Since these games often involve large numbers of players and complex interactions, algorithms must be developed that can find these solutions. The optimal solution that Algorithmic Game Theory seeks to find is one in which no player has an incentive to change their strategy.

Chapter \ref{cha:gameDefinitions} provides examples to make it easier to understand how algorithmic games work. This chapter introduces some well-known examples of algorithmic games, such as the Prisoner's Dilemma \ref{sec:prisonersDilemma}. These examples serve as an introduction to how these types of games work and to the different types of games in Algorithmic Game Theory. Other types of games discussed are Coordination Games \ref{sec:coordinationGames} and Mixed Strategy Games \ref{sec:mixedStrategyGames}. Suggestions on how to formally define such games are given in section \ref{sec:formalDefinintion}.

In chapter \ref{cha:solutionStrategies} different strategies to solve these algorithmic games are discussed. These solution strategies introduce the concept of Nash equilibria. The different strategies and types of Nash equilibria are explained using simple examples. In \ref{sec:learningGames} the concept of best responses for learning games is introduced. The topics of Bayesian and Cooperative Games are also briefly discussed in section \ref{sec:bayesianGames} and \ref{sec:cooperativeGames}.

The complexity of computing Nash equilibria is explained in chapter \ref{cha:complexityOfFindingNash}. Here the PPAD-completeness of Nash is discussed and the Lemke-Howson algorithm for computing Nash is introduced in section \ref{sec:lemkeHowson}.

Chapter \ref{cha:summary} provides a summary of the topic and an overview of the areas that are currently being researched. Interesting new applications and the importance of Algorithmic Game Theory in today's world are discussed.

\section{Motivation}
Algorithmic Game Theory can be applied to many different areas, such as markets. Here Algorithmic Game Theory is used to understand the effects of existing prices, production, consumption, introduction of new products or changes in taxes. It can be used to ensure price stability and parity between supply and demand of products. Nash equilibria can be used to calculate the optimal price for a product (the price at which the supply and demand curves intersect).

Another area in which Algorithmic Game Theory can be used is finance. It can be used to design algorithms for high-frequency trading, in which computers automatically execute trades at high frequencies based on complex rules.

Other possible applications include modeling of interaction and coordination in a cloud or distributed system. Robotics, network security and resource management are also influenced by Algorithmic Game Theory.

The above mentioned areas are only examples for applications of Game Theory. There are many more areas in which Algorithmic Game Theory can be used to for example model or improve systems. This is why Algorithmic Game Theory is becoming more and more important since the introduction of the Internet, as such systems (like trading systems or online markets) will only continue to grow in number, complexity and importance.

\section{Related Work}
This paper is mainly based on the first two chapters of the book "Algorithmic Game Theory" by Nisan et al. \autocite{agt}. This paper serves as an introduction to the basic principles of Algorithmic Game Theory and the more advanced concepts covered in the book. The book provides a very thorough overview of the topic and contains many more in depth explanations on things like graphical games or various algorithms and their design that are not mentioned in this paper. 