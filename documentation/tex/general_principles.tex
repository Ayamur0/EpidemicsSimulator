The models used by the the app developed in this work are based on chapters 19-21 of the book
"Networks Crowds and Markets" by Easley and Kleinberg \cite{networks}.

\section{Simplest model}
The first model proposed by Easley and Kleinberg \cite{networks} uses a very simplistic
representation of networks. The network is represented as a tree, with each layer representing
the nodes that come into contact with infected nodes from the previous cycle. The root
of the tree is the first person to contract the disease in the social network. During the
first cycle the $k$ nodes at a depth of 1 may or may not get infected with the disease, depending
on the infectiousness of the disease. During the second cycle, each of these $k$ nodes
now comes into contact with $k$ other nodes at a depth of 2. This means during the second
cycle, $k \cdot k = k^2$ people are potentially at risk of infection. During each cycle, the
number of people who come into contact with the disease increases by a factor of $k$, for
a total of $k^{cycle}$ people potentially exposed to the disease during each cycle.
Figure \ref{fig:tree_network} shows a possible network structure.

\begin{figure}
    \centering
    \includegraphics[width=0.75\linewidth]{images/network_tree.png}
    \caption{Tree representation of a network showing the spread of a disease (source: \cite{networks})}
    \label{fig:tree_network}
\end{figure}

The book \cite{networks} also explains the concept of the reproductive number $R_0$ in
relation to this network. $R_0$ is the expected number of new cases of the disease
caused by a single infected individual. In the case of the tree network, this means 
$R_0 = pk$, where $p$ is the infectiousness of the disease. If $R_0 > 1$, the number
of cases will increase over time because each person infects more than one other person
on average. Thus, there is a possibility that the disease will never
die out. If $R_0 < 1$, the number of cases will decrease because on average each person will infect less
than one other person, resulting in the disease dying out in a finite number of cycles.
With this knowledge, the importance of $R_0$ in controlling an epidemic is clear. To prevent
or stop an epidemic, the $R_0$ factor of a disease has to be less than 1.

\subsection{Limitations}
This model has several limitations. It assumes each person has contact with the same number of
people, which is never the case in real social networks. There are always people who come into
contact with more people than others. Also, each person can only infect others during the
first cycle after being infected. It is not possible for a person to infect others during multiple cycles for longer lasting diseases. Further, it is not possible for a person
to become infected a second time because there are no loops within the tree.

\section{SIR Model}
The SIR Model is a more advanced model that allows modeling of most social network structures
by generalizing the contact structure. Easley and Kleinberg \cite{networks} define three 
stages each node can have:
\begin{itemize}
    \item \textbf{S}usceptible: Node is not yet infected but susceptible to infection from its neighbors
    \item \textbf{I}nfectious: Node has caught the disease and has a probability to infect its neighbors
    \item \textbf{R}emoved: The node went through the full infection period and is removed from consideration for future cycles
\end{itemize}
The SIR Model uses a directed graph do indicate which nodes are neighbors and thus susceptible 
to infecting each other. The resulting graph does not have to be anti-symmetric it may also contain
undirected edges. 

In addition to the network the SIR Model uses two additional quantities to control the 
epidemic: $p$ the probability an infected node infects a susceptible node and $t_I$ the length
of the infection period.

Initially some nodes are in the $I$ state while all others are in the $S$ state. After
each cycle all neighbors of the nodes in the $I$ state are infected with a probability of $p$.
After $t_I$ steps a node is removed from the $I$ state and can no longer infect others or be
infected by others.

\subsection{Reproductive Number}
\label{sub:r0}
For these more complex networks calculating the reproductive number $R_0$ is not as trivial
as for a tree based network. The book contains an example for a network shown in figure \ref{fig:narrow_network}
in which even highly contagious diseases will die out.

\begin{figure}
    \centering
    \includegraphics[width=0.5\linewidth]{images/narrow_network.png}
    \caption{A network where the disease must pass through a narrow structure of nodes. (source: \cite{networks})}
    \label{fig:narrow_network}
\end{figure}

Let $d$ be a disease with a high contagiousness of $p=0.8$ and an infection period of $t_I=1$. 
Using the assumptions made for calculating $R_0$ in the tree network, this would result 
in $R_0 = 2 \cdot 0.8 = 1.6$, indicating that the disease is relatively likely 
never to die out. However, due to the structure of the network, the disease is going to 
die out relatively quickly. The probability of the disease not spreading during a cycle is
$0.2^4=0.0016$, so on average the diseases will die out after $\frac{1}{0.0016}=625$ cycles.
This shows that in more complex networks, the structure of the network also plays a big role
in the progression of the epidemic. It is not possible to calculate $R_0$ from the
characteristics of the disease alone, the network structure must also be taken into account.
Knowing this, it is impossible to accurately calculate $R_0$ in a highly complex network.
It is however possible to observe the value of $R_0$ during the epidemic for previous cycles.
Let $R_0^k$ be the $R_0$ value for cycle $k$ and $I_k$ the number of infected nodes in cycle $k$.
Then $R_0^k = \frac{I_k}{I_{k-1}}$, with this the evolution of the $R_0$ value can be 
observed, making it possible to estimate $R_0$ for future cycles and helping to understand
whether the disease is currently dying out ($R_0<0$) or not ($R_0>1$). A simulation can also
help to understand how the $R_0$ value of diseases will behave for a disease with certain 
quantities in a complex network.

\subsection{Limitations}
There are still some limitations to the SIR Model. Each person can only catch the disease at
most once and the model does not allow for simulating multiple diseases at the same time.
However, the model allows most social networks to be represented, making it relatively easy
to extend this network to include more complex diseases, e.g. with different levels of contagiousness
depending on the time a node has been infected or a non-constant infection period.

\section{SIS Model}
The SIS Model is an extension to the SIR Model that also allows nodes to be reinfected multiple
times. The \textbf{R}emoved state is exchanged with the \textbf{S}usceptible state, so nodes
that are removed from the \textbf{I}nfected state are placed back in the \textbf{S}usceptible state.
With this change, it is possible for diseases to survive an infinite amount of time in an SIS Model, as opposed to an SIR Model, where each simulation ends after a finite number of steps after the 
disease has burned through all nodes. The only way for the simulation in a SIS Model to end is
if all infected nodes fail to transmit the disease to any of their neighbors $t_I$ times.

\section{Custom Model}
\label{sec:custom_model}
\subsection{Parameters}
The model used by the developed app is a further extension of the SIR and SIS Models. It 
combines the SIR and SIS Models by moving nodes that finish the infection period either
into the \textbf{R}emoved or \textbf{S}usceptible state depending on a quantity $f$. Each
disease has a mortality rate $f$ that determines whether a node is moved into the 
\textbf{R}emoved state and considered dead or moved back into the \textbf{S}usceptible
state if it survived the infection. The \textbf{S}usceptible state is split into two
substates: nodes that have never been infected before and nodes that have been put back into it after being infected at least once. This allows for different infectiousness values $p_I$ for the
first infection of a node and $p_r$ for reinfections. 

The $t_I$ parameter will be split into two new parameters: $t_{min}$ the minimum time an infection lasts before the node can be cured and $t_\rho$ the probability a node is cured from its infection
after the minimum time $t_{min}$ has elapsed. This results in four states so far:

\begin{itemize}
    \item Healthy: Susceptible nodes that were never infected before. They are at risk of being
    infected by their neighbors with a probability of $p_I$
    \item Infected: Nodes that are currently infected. The infection duration is least $t_{min}$ cycles and
    the exact duration is determined by the probability $t_\rho$. After the infection ended the
    node is moved into the cured state with probability $1-f$ or into the deceased state with
    probability $f$.
    \item Cured: Nodes that were infected but survived. They are at risk of being
    infected by their neighbors with a probability of $p_r$ 
    \item Deceased: Nodes that died from the infection. They are not considered in future cycles
    and can not infect others anymore.
\end{itemize}

The infectiousness of a disease usually changes over the course of the infection. In most
cases, an infected person is most likely to infect others during the first few days of contracting
the disease. This should also be represented in the custom model. Thus, the values of $p_I$ and $p_r$ need to change with the time $t_c$ for which a node has been infected. These time-dependent probabilities
will be called $p_I^t$ and $p_r^t$. A healthy node now has a probability of $p_I^t$ to
be infected by another node that has been infected since $t$ cycles. Analogously, a cured node 
now has a probability of $p_r^t$.

Another part of epidemics that is not represented in the SIR or SIS Model are vaccines.
During the course of an epidemic, vaccines can be used to reduce the fatality rate
for infected persons and reduce the likelihood that vaccinated persons will contract the disease. To accommodate this in the custom model, two new node states are added:
\begin{itemize}
    \item Vaccinated: Nodes that are vaccinated and now have a probability of getting infected
    of $p_v^t$ and a fatality rate of $f_v$.
    \item Unvaccinated: Nodes that are not vaccinated and use the above mentioned probabilities
    $p_I^t$/$p_r^t$ and $f$.
\end{itemize}
These two new states are not exclusive with the previous four states. Each node simultaneously
has one of the two states, vaccinated or unvaccinated, and one of the previous four states 
healthy, cured, infected or deceased.

\subsection{Network model}
The network model is very similar to the one used by the SIR and SIS Model. However it does
not allow for directed connections as there are very limited uses for those connections. 
Almost every human contact is bidirectional if for example one person gets close enough to
another person to contract an air transmitted disease this transmission can always happen
in both directions.

The network model used in this custom model organizes the nodes of one social circle in groups
to more clearly organize the network for more visual clarity. Each group can be 
considered a small-world contact network were the nodes in each group are a localized part
of the network that is highly connected with fewer connections to other groups.

\section{Characteristics of epidemics}
\subsection{Oscillating diseases}
As Easley and Kleinberg \cite{networks} explain, diseases with certain characteristics
can cause an oscillating number of infections. Consider a network with several highly
connected groups that have have few connections between the groups. If a disease
with a very high infection probability $p\geq0.9$, a period of immunity $i > 0$ and a
fatality rate close to zero breaks out in such a network, the umber of infections will oscillate.
An example of how the number of infections might evolve over time can be seen in figure 
\ref{fig:oscillation}.
\begin{figure}[!ht]
    \centering
    \includegraphics[width=0.75\linewidth]{images/oscillation.png}
    \caption{Amount of infections over time in a network of highly connected 
    groups. The disease has a high infection rate and low fatality. (source: \cite{networks})}
    \label{fig:oscillation}
\end{figure}

This happens because due to the high infectiousness of the disease, almost all nodes in 
a group that has at least one infected node will be infected within a few cycles. Since
almost all nodes are infected after only a few cycles, there are only a few new infections 
in the next cycles due to the sparse set of available healthy nodes. If the initial
wave started at cycle $c$, then at cycle $c' = c + t_i + i$ the nodes of the first wave 
are susceptible to infection again. This drastically increases the number of possible targets for new infections, which in turn increases the number of new infections again. This results in 
the oscillation of the number of new infections.

Such a scenario can be modeled with the created tool. The network consists of 5 groups with 2,000 nodes each. The groups have a lot of intra-group connections (5 per node) and only a small number of connections to the
other groups. Each group is connected to two other groups with 1 edge per node and it is ensured that
all groups have a path to all other groups.

The disease that will be simulated in this network has an infection rate of 0.2, a fatality
rate of 0, a infection duration of 5 cycles and a immunity period of 3 cycles. 
\clearpage For simplicity the reinfection rate is the same as the initial infection rate and no vaccinations are used.

\begin{figure*}
    \centering
    \begin{subfigure}[b]{0.475\textwidth}
        \centering
        \includegraphics[width=\textwidth]{images/oscillation0.png}
        \caption[Network2]%
        {{\small Network after 0 Steps \newline}}    
        \label{fig:mean and std of net14}
    \end{subfigure}
    \hfill
    \begin{subfigure}[b]{0.475\textwidth}  
        \centering 
        \includegraphics[width=\textwidth]{images/oscillation12.png}
        \caption[]%
        {{\small Network after 12 Steps, infections are at a maximum}}    
        \label{fig:mean and std of net24}
    \end{subfigure}
    \vskip\baselineskip
    \begin{subfigure}[b]{0.475\textwidth}   
        \centering 
        \includegraphics[width=\textwidth]{images/oscillation18.png}
        \caption[]%
        {{\small Network after 18 Steps, infection are at a minimum}}    
        \label{fig:mean and std of net34}
    \end{subfigure}
    \hfill
    \begin{subfigure}[b]{0.475\textwidth}   
        \centering 
        \includegraphics[width=\textwidth]{images/oscillation24.png}
        \caption[]%
        {{\small Network after 24 Steps, infection are at a maximum again}}    
        \label{fig:mean and std of net44}
    \end{subfigure}
    \caption[ State of the network showing the oscillation of infected nodes ]
    {\small State of the network showing the oscillation of infected nodes. Red nodes are currently infected, dark green ones have never been infected and 
    light green ones were previously infected but have recovered and have immunity.} 
    \label{fig:osc_network}
\end{figure*}
\clearpage

Initially, 2 random nodes are infected with the disease. Figure \ref{fig:osc_network}
shows the state of the network after 0, 12, 18 and 24 cycles.
The number of infections over time can be seen in figure \ref{fig:oscillation_p02}
which clearly shows the oscillatory nature of this disease. Over time, the amplitude of the
waves decreases, as the number of people infected at the same time and thus getting cured
at the same time decreases, so the infections are no longer synchronized and each cycle has the same number of new nodes available to be infected. This happens because the
infection rate was too low. The disease was not able to explosively infect all new nodes
as soon as they became available, so some nodes remained uninfected for 2-3 cycles,
shifting their cure time and breaking the oscillation. Increasing the infection rate
to $p = 0.3$ prevents this, as the disease now spreads fast enough to infect every
available node immediately. The result of this is shown in figure \ref{fig:oscillation_p03}.

\begin{figure}[!ht]
    \centering
    \includegraphics[width=0.75\linewidth]{images/oscillation_infections.png}
    \caption{Amount of infections over time showing the oscillating nature with $p = 0.2$}
    \label{fig:oscillation_p02}
\end{figure}

\begin{figure}[!ht]
    \centering
    \includegraphics[width=0.75\linewidth]{images/oscillation_infections3.png}
    \caption{Amount of infections over time showing the oscillating nature with $p = 0.3$}
    \label{fig:oscillation_p03}
\end{figure}
\clearpage

Now the same network is used but the infection rate of the disease is decreased to 0.1.
Because the disease is now not spreading as explosively as before it always has enough targets
to infect until the previously infected nodes become cured again. Thus the amount of new 
infections is more consistent and does not oscillate as seen in figure \ref{fig:no_oscillation}.

\begin{figure}[!ht]
    \centering
    \includegraphics[width=0.75\linewidth]{images/no_oscillation.png}
    \caption{Amount of infections over time showing that no oscillations occur if with $p = 0.1$}
    \label{fig:no_oscillation}
\end{figure}

Another way to break the oscillation is to keep $p=0.3$ but decrease the duration of the infection to 2 cycles 
and remove the immunity period. Now the infected nodes become cured so fast that there are
enough new nodes to infect for every cycle, again resulting in a relatively consistent
amount of new infections as indicated in figure \ref{fig:no_oscillation2}

\begin{figure}[!ht]
    \centering
    \includegraphics[width=0.75\linewidth]{images/no_oscillation2.png}
    \caption{Amount of infections over time showing that no oscillations occur even with $p = 0.3$ if the duration is too short}
    \label{fig:no_oscillation2}
\end{figure}
\clearpage

\subsection{Epidemics without diseases}
Diseases are not the only thing that is capable of spreading through social networks. As
previously mentioned computer viruses spread in a similar pattern. However, there are other
completely different things that can spread in social networks in a similar way.
Information, for example, can be modeled using a very similar approach to the one proposed
for diseases. Information can be passed from person to person whenever they talk to each other,
be it over a long distance through things like text messages or by meeting in person.

Other things that can spread through social networks include traditions or certain habits.
Looking at the world, people within each country are much more likely to interact with other people from the same country than with people from other countries. This means each
country can be modeled as a highly connected group of people with few connections to other
countries. This model can be used to explain how each country has its own traditions that
differ from other countries the further away they are. Traditions spread easily within 
a country, but the further away a country is, the less likely the tradition is to spread to that country.
Because there are few connections to other countries, foreign traditions rarely spread to other countries and if they do, they are drowned out by the explosively spreading regional traditions of the other country.

These models can also be used to model genetic inheritance. As suggested by
Easley and Kleinberg \cite{networks} this can be done by connecting parents to their offspring.
Simulations can then be run to show the spreading of various genetic factors through
the ancestry trees. This type of simulation itself offers many areas that can be expanded
to create various simulations relevant to understanding human history and origins. 
It can also be used to indicate and visualize the existence of common ancestors and 
can give an idea of how many generations it takes to reach them. One such model that deals
with genetics and common ancestors is the Wright-Fisher Model. Haller and Messer \cite{genetics}
go into more detail on how this model and the genetic simulation work. Genetic simulation 
and the Wright-Fisher Model are a large research area in themselves, there are several other
works that discuss these topics, explain the mathematics behind those models, 
possible uses etc. \cite{genetics2} \cite{genetics3} \cite{genetics4}.


These different applications shows how the use of epidemics and social networks goes way beyond modeling diseases and can help understand complex relations in various different areas of research.