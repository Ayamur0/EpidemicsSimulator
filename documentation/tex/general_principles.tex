The models the app developed in this work uses are based on chapters 19-21 of the book
"Networks Crowds and Markets" Easley and Kleinberg \cite{networks}.

\section{Simplest model}
The first model proposed by Easley and Kleinberg \cite{networks} uses a very simplistic
representation of networks. The network is represented as a tree with each layer representing
the nodes who come into contact with infected nodes form the previous cycle. The root
of the tree is the first person to contract the disease in the social network. During the
first cycle the $k$ nodes at a depth of 1 may or may not get infected with the disease based
on the infectiousness of the disease. During the second cycle each of these $k$ nodes
now comes into contact with $k$ other nodes at a depth of 2. This means during the second
cycle $k \cdot k = k^2$ people are potentially at risk of infection. During each cycle the
amount of people that come into contact with the disease increases by a factor of $k$ for
a total of $k^{cycle}$ that potentially come into contact with the disease during each cycle.
Figure \ref{fig:tree_network} shows a possible network structure.

\begin{figure}
    \centering
    \includegraphics[width=0.5\linewidth]{images/tree_network.png}
    \caption{Tree representation of a network showing the spread of a disease (source: \cite{networks})}
    \label{fig:tree_network}
\end{figure}

The book \cite{networks} also explains the concept of the reproductive number $R_0$ in
relation to this network. $R_0$ is the expected number of new cases of the disease
caused by a single infected individual. In the case of the tree network this means 
$R_0 = pk$ with $p$ being the infectiousness of the disease. If $R_0 > 1$ the amount
of cases will increase over time because each person infects more than one other person
on average. Thus there is a possibility that the disease will never
die out. If $R_0 < 1$ the amount of cases is decreasing on average each person infects less
than one other person which leads to the disease dying out in a finite number of cycles.
With this knowledge the importance of $R_0$ in fighting an epidemic is clear. To prevent
or stop a epidemic the $R_0$ factor of a disease has to be below 0.

\subsection{Limitations}
This model has several limitations. It assumes each person has contact to the same amount of
people which is never the case in real social networks. There are always people who come into
contact with more people than others. Also each person can only infect others during the
first cycle after they got infected. It does not allow for a person to infect others during
multiple cycles for longer lasting diseases. Further it is not possible for a person
to get infected a second time as there are no loops withing the tree.

\section{SIR Model}

