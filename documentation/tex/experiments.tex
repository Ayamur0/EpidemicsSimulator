In this chapter some experiments will be conducted. These experiments are modeled
after the scenarios described in the Theory chapter %TODO ref
The goal is to recreate these situation using the developed tool and running a
simulation with the described parameters. These simulations can then be used
to support or refute the assumptions that were made on how the diseases
theoretically should behave in those scenarios.

\section{Importance of $R_0$}
The role of the reproductive number $R_0$ was discussed in section \ref{sub:r0}.
It is the deciding factor for whether a disease is dying out ($R_0 < 1$) 
or able to live indefinitely ($R_0 > 1$). Calculating $R_0$ in complex social
networks is very hard because the structure of the networks has a big impact on $R_0$
in addition to the characteristics of the disease. For these experiments a very
rough estimation of $R_0=e_\mu \cdot p$ is used as this is sufficient for estimating whether
the disease should die out in the experiment or life for a long time.
$e_\mu$ is the average amount of connection per node, let $N$ be the total
amount of nodes in the network and $E$ the total amount of edges then
$e_\mu=\frac{E}{N}$. $p$ is the probability for a node to get infected
if one of its neighbors has the disease.

\subsection{The network}
The network that is used in these experiments contains 3 groups with 5000 nodes
each. Group 1 has 5 intra group connections for each node, group 2 has 4 and
group 3 has 3 intra group connections. Between each pair of groups there are
2 connections per node. The resulting network can be seen in figure %TODO
It contains a total of 15,000 nodes. Let $N_i$ be the amount of nodes in group $i$
then the amount of edges can be calculated using the following equation:
\begin{equation}
    E = \frac{n_1}{2} * 5 + \frac{n_2}{2} * 4 + \frac{n_3}{2} * 3 + \frac{n_1+n_2}{2} * 2 + \frac{n_1+n_3}{2} * 2 + \frac{n_2+n_3}{2} * 2
\end{equation}
Using that equation the it can be calculated that the network contains 60,000 edges.
This leads to an average amount of connections per node of $e\mu=\frac{60,000}{15,000} = 4$.

\subsection{Experiment with $R_0 < 1$}
Since this experiment will use the same network as the one for a $R_0 > 1$ the
facor $e_\mu$ is static and cannot be changed. This makes $p$ the only deciding
factor for $R_0$ it can be calculated using $p = \frac{R_0}{e_\mu}$ so in this case
with $R_0 < 1$, $p < \frac{1}{4}$. For cases with $R_0$ close to 1 there is still
a decent chance that the disease will live for a long time which is why $p=0.125$
is chosen for a estimated $R_0=0.5$ to ensure that the disease dies out in a finite
amount of steps.

The parameters for the disease of this experiment then are:
\begin{itemize}
    \item %TODO
\end{itemize}

Running the simulation shows that the disease is never able to spread to a large
amount of people and dies out after only \~ %TODO
steps. This can also be seen in the graph %TODO 
which shows the amount of new infections per cycle. 
Increasing the initial amount of infected nodes to 5,000 shows that even with
such a large amount of infections the disease dies out after a certain amount of steps.
In this case it took %TODO
steps, graph %TODO
shows the amount of new infections up to that point.

This experiment supports the theory that for a $R_0 < 1$ the disease will die out
in a finite amount of steps.

\subsection{Experiment with $R_0 > 1$}
This network uses the same network as the previous one, so $p$ is again the deciding
factor for $R_0$. $p = \frac{R_0}{e_\mu}$ so for this case with $R_0 > 1$, $p > \frac{1}{4}$
since the closer $R_0$ to 1 the higher the chance the disease dies out $p=0.375$ is chosen
resulting in a $R_0 = 1.5$.

The properties that were changed from the previous experiment are:
\begin{itemize}
    \item %TODO
\end{itemize}

After running the simulation for 2,000 cycles the disease is still alive and infecting
new nodes. This supports the theory that with a $R_0 > 1$ there is a probability $> 0$
that the disease never dies out. Figure %TODO
shows the amount of new infections which is remains relatively constant 
from cycle x to %TODO.

\subsubsection{Changing the network}
To show that the network structure indeed has an impact on $R_0$ this experiment
uses the same disease as the previous one. The network will have only half as many
edges, group 1 has 2 intra group connections for each node, group 2 has 2 and
group 3 has 2 intra group connections. Between each pair of groups there are
1 connections per node. This new network has 30,000 edges resulting in $e_\mu$ = 2.
Thus the new estimated $R_0=2\cdot0.375=0.75$ which is smaller than 1 so they
expectation is that the disease will die out even though it has the same infection 
rate as before.

After %TODO
cycles the disease has died out which supports the theory that the network structure
has an impact on the spreading of diseases along with the characteristics of the disease.
Graph %TODO 
shows the amount of infections until the disease died out.

\section{Small World Phenomenon}
%TODO allgemeiner text

\subsection{Experiment}
For this experiment a network with only one group is created. That group
contains 10,000 nodes with each node having %TODO
connections. To show that each node can be reached in only %TODO
steps, a disease with an infection rate of 1 and infection duration of
1000 is used. Initially only one
node is infected. Figure %TODO shows the network after x cycles.
After cycle %TODO
all nodes are infected which means they were all able to be reached from the 
starting node in only %TODO
steps.


