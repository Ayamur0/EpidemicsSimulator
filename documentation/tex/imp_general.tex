An app that allows for the simulation of epidemics is created. First the required functionalities
of the app need to be determined. It must be able to simulate different epidemics cases for
which different social networks and diseases are required.

\section{Network Editor}
The app needs an editor that allows for the creation of different social networks. Since
social networks can consist of a huge amount of people this editor needs to allow the user
to quickly create networks with a high amount of nodes and connections. For this there
will be three settings for a group of nodes:

\begin{itemize}
    \item \textbf{Size}: The amount of nodes in the group
    \item \textbf{Intra group connections}: The amount of edges each node has to other nodes
    of the same group
    \item \textbf{Intra group connections delta}: The variation for the amount of edges each
    node has ot other nodes of the same group. Eg. a connection amount of 5 with a delta of 3
    would result in each node having between 2 and 8 connections.
\end{itemize}

A network then consists of any amount of groups each with different settings. To allow
for connections between these groups there are similar settings availabe for each pair of
groups:

\begin{itemize}
    \item \textbf{Inter group connections}: The amount of edges each node has to other nodes
    of the other group
    \item \textbf{Inter group connections delta}: The variation for the amount of edges each
    node has ot other nodes of the other group. Eg. a connection amount of 5 with a delta of 3
    would result in each node having between 2 and 8 connections.
\end{itemize}

Using these properties the user is able to quickly generate big networks to model most
social situations. To visualize the structure of the network there will also be a 3D view
of the current network which can be updated after each change. Chapters \ref{cha:network_generation}
and \ref{cha:network_display} explain the implementation of this feature in more detail.

\section{Disease Editor}
To create the diseases that spread in the network the app has a tab were the required properties
for each diseases can be set. These include the properties discussed for the custom model in
section \ref{sec:custom_model} as well as some other properties required for visualizing the network:
\begin{itemize}
    \item \textbf{Name}: Name of the disease, will be shown in the legend when displaying the network
    \item \textbf{Color}: Color nodes infected with this disease will have
    \item \textbf{Fatality rate}: $f$, the chance a node is moved into the deceased state after the infection period is over
    \item \textbf{Vaccinated fatality rate}: Fatality rate for vaccinated nodes
    \item \textbf{Infection rate}: $p_I$, the chance an unvaccinated node gets infected by a neighbor
    \item \textbf{Reinfection rate}: $p_r$, the chance a previously infected node gets infected again by a neighbor
    \item \textbf{Vaccinated infection rate}: $p_v$, the chance a vaccinated node gets infected by a neighbor
    \item \textbf{Minimum duration}: $t_{min}$, the minimum cycles an infection lasts
    \item \textbf{Cure chance}: $t_\rho$, chance a node gets cured (or dies) each cycle after it has been
    infected for $t_{min}$ cycles
    \item \textbf{Immunity period}: Amount of cycles a node is immune after being cured
    \item \textbf{Infectiousness factor}: Decrease of infection rates with each cycle the node has been infected.
    Eg. with a initial $p_I = 0.5$ and a factor $I = 0.9$ the infection rate after $x$ cycles is $p = p_I \cdot I^x$.
    \item \textbf{Initial infections}: Amount of nodes infected with this disease in cycle 0, the start of the simulation
\end{itemize}

\section{Simulation}
To simulate diseases the app contains two different tabs. One where the network is displayed
in 3D and the current state of each node is indicated by colors. The other tab only contains
text for each group showing how many nodes are infected, deceased, etc. per group. The simulation
without visually seeing the graph allows for faster simulation of a high amount of steps and for
networks with a high amount of nodes (over 100,000, depends on computing power of the system)
building the graph might take a long time making the simulation with the visual representation
almost unusable.

Both simulations contain buttons to advance one step, automatically advance steps over time,
reset the simulation and save the statistics collected during the simulation. The simulation
which displays the networks also contains buttons to alter the display of the network, eg. to hide
certain node groups or edges to allow viewing only the areas of the network the user wants to observe.
Chapter \ref{cha:visual_simulation} explains the implementation of this feature in more detail.

\section{Statistics}
The last tab of the app can be used to view the statistics collected during simulations.
It displays the individual stats in a coordinate system and contains functions to split
the stats by different parameters, eg. show only infections with a certain disease or all
diseases or of only a certain group. The graph can be viewed as the individual value of each
step as well as the cumulative value up to each step.

\section{Frameworks}
The app is constructed using python. For the UI QT \cite{qt} is used, this is further explained
in chapter %TODO
For the display of the graphs the plotly \cite{plotly} is used along with the dash implementation
it provides for creating webpages. The graphs will then be displayed on the dash webpage
which is embedded in the QT app. The implementation of the website is explained in chapter %TODO.
For saving a project the JSON file format is used. The saved project files contain all 
necessary information, like the groups of the network, diseases and stats of previous simulations
to allow closing the app after saving without losing any progress.
