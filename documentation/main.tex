
\documentclass[oneside]{ausarbeitung}
\bibliography{latexlit}


% ----------------------------------------------------------------------

\begin{document}

%--- Sprachauswahl
% Erlaubte Werte:
%   \selectlanguage{english}
%   \selectlanguage{ngerman}
\selectlanguage{english}

%--- Art der Arbeit
% Erlaubte Werte:
%   \Praxissemesterbericht
\Projektbericht
%   \Bachelorarbeit
%   \Seminararbeit
%   \Masterarbeit

%--- Studiengang:
% Erlaubte Werte:
%   \Informatik
%   \Elektronik
%   \DataScience
\Informatik

\title{Epidemics}

\author{Sascha Mößle, Tim Staudenmaier}
\matrikelnr{75987, 75981}

%--- Ist der Erstbetreuer (\examinerA) an der Hochschule ein Professor?
% Erlaubte Werte:
%   \examinerIsAProfessortrue   % Ja
%   \examinerIsAProfessorfalse  % Nein
\examinerIsAProfessortrue   % Ja

%--- Betreuer
\examinerA{Prof.~Dr.~Thomas~Thierauf}
%\examinerB{Prof.~Dr.~Ulrich~Klauck}

%--- Einreichungsdatum
\date{14.3.2023}

%--- Angaben zur Firma
% Auskommentieren, wenn die Arbeit nicht bei einer ext. Firma gemacht wurde.
%\companyname{Beispielfirma}
%\industrialsector{Beispielbranche}
%\department{Beispielabteilung}
%\companystreet{Beispielstr. 1}
%\companycity{12345 Musterstadt}

%--- Angaben zum Betreuer bei dieser Firma
%\advisorname{Name des Betreuers}
%\advisorphone{(01234) 567-890}
%\advisoremail{name@company.xxx}

%--- Titelseite Anzeigen
\maketitle
\cleardoublepage

%---
\pagenumbering{roman}
\setcounter{page}{1}

%--- Firmendaten Anzeigen
% Auskommentieren, wenn die Arbeit nicht bei einer ext. Firma gemacht wurde.
%\makeworkplace
%\cleardoublepage

%--- Eidesstattliche Erklärung anzeigen
\makeaffirmation
\cleardoublepage

%---
\begin{abstract}
 Epidemic diseases are an important area of studies. During the history of humans there
 were multiple instances of epidemics singificantly affecting large parts of the world.
 One recent example would be the Corona virus which brought most parts of the world to a 
 standstill. The area of epidemics focuses on such contagious diseases which spread from
 person to person, like Corona of influenza.

 Depending on the characteristics of the virus epidemics can have a very different progression.
 Some may explosively affect the whole world while not causing many casualties while others
 spread very slowly but linger for a long time with a high fatality rate. An important 
 part of studying these diseases is simulating how different diseases could affect the world.
 The simulation of such diseases can be accomplished using network graphs. This paper introduces
 a model wich can be used for simulation and explains how the program that allows to simulate
 a disease using that model is built.
\end{abstract}
%-----------------------------------------------------------------------
\cleardoublepage
\tableofcontents

%---
\listoffigures

%---


\cleardoublepage
\pagenumbering{arabic}
\setcounter{page}{1}

\algrenewcommand\algorithmicrequire{\textbf{Input:}}
\algrenewcommand\algorithmicensure{\textbf{Output:}}

% ----------------------------------------------------------------------

\chapter{Introduction}
\label{cha:introduction}
\section{Motivation}
An epidemic outbreak can have a major impact on the world. Past epidemics have shown that
if we do not know how to respond to an epidemic outbreak and are not prepared for 
such cases, a new disease can wipe out large parts of the world. The black
death caused the death of about 30\% to 60\% of all Europeans (75-200 million)
during the 1300s \cite{blackDeath}.
To better understand such scenarios, simulations play an important role, as the study of real
cases of epidemic outbreaks is difficult since there are only so many in the history of humans.
Also, in the event of an outbreak, it is important to be able to simulate the next few days/weeks
in order to accurately predict how the epidemic will evolve.

For this reason this work will discuss an approach to simulate such epidemics by modeling
networks of people and then simulating the spreading of diseases with different characteristics
in these networks.

\section{Problem definition}
A model of the social network in which the disease is spreading in is crucial to the simulation.
Depending on the transmission method of the disease, this network may be highly connected in 
the case of a disease with airborne transmission or have only few connections for diseases
that are sexually transmitted. In addition, different diseases can spread
completely different in the same social network even if they have the same transmission method
because the characteristics of the disease also play an important role in how it spreads. 
As suggested by Easley and Kleinberg \cite{networks}
the transmission of computer viruses works in a similar way can therefore also be modeled
using networks.

The networks that can be created using the app must be able to model different social
networks. The most important part for the epidemics simulation is the modeling of the amount
of contacts with other people each person has. Since each group can have a significant number
of members an efficient method for creating networks with large amounts of nodes is required that
still allows to model most social networks.

A method to visualize these networks in a clear way is needed. The visual 
representation must still be usable with large amounts of nodes (e.g. over 100,000 nodes).
To make the visualization of the network more usable, some settings must be provided
to modify the displayed network, such as hiding certain connections or nodes. It also needs
ways to represent the current state of the network in respect to the spread of the diseases.

The app also needs to allow for creation of multiple diseases with different characteristics.
To simulate various epidemic scenarios properties like the infectiousness, duration of illness
or fatality need to be editable. 

The app should be able to simulate multiple diseases at the same time. The simulation needs
to take into account which humans have contact with each other and then simulate the spreading
of the diseases according to the characteristics of each disease and group of people.

During the simulation the app will collect statistics that allow a review of key information
after the simulation, such as the number of new infections over time.

\chapter{Theory of modeling epidemics}
\label{cha:general_principles}
The models used by the the app developed in this work are based on chapters 19-21 of the book
"Networks Crowds and Markets" by Easley and Kleinberg \cite{networks}.

\section{Simplest model}
The first model proposed by Easley and Kleinberg \cite{networks} uses a very simplistic
representation of networks. The network is represented as a tree, with each layer representing
the nodes that come into contact with infected nodes from the previous cycle. The root
of the tree is the first person to contract the disease in the social network. During the
first cycle the $k$ nodes at a depth of 1 may or may not get infected with the disease, depending
on the infectiousness of the disease. During the second cycle, each of these $k$ nodes
now comes into contact with $k$ other nodes at a depth of 2. This means during the second
cycle, $k \cdot k = k^2$ people are potentially at risk of infection. During each cycle, the
number of people who come into contact with the disease increases by a factor of $k$, for
a total of $k^{cycle}$ people potentially exposed to the disease during each cycle.
Figure \ref{fig:tree_network} shows a possible network structure.

\begin{figure}
    \centering
    \includegraphics[width=0.75\linewidth]{images/network_tree.png}
    \caption{Tree representation of a network showing the spread of a disease (source: \cite{networks})}
    \label{fig:tree_network}
\end{figure}

The book \cite{networks} also explains the concept of the reproductive number $R_0$ in
relation to this network. $R_0$ is the expected number of new cases of the disease
caused by a single infected individual. In the case of the tree network, this means 
$R_0 = pk$, where $p$ is the infectiousness of the disease. If $R_0 > 1$, the number
of cases will increase over time because each person infects more than one other person
on average. Thus, there is a possibility that the disease will never
die out. If $R_0 < 1$, the number of cases will decrease because on average each person will infect less
than one other person, resulting in the disease dying out in a finite number of cycles.
With this knowledge, the importance of $R_0$ in controlling an epidemic is clear. To prevent
or stop an epidemic, the $R_0$ factor of a disease has to be less than 1.

\subsection{Limitations}
This model has several limitations. It assumes each person has contact with the same number of
people, which is never the case in real social networks. There are always people who come into
contact with more people than others. Also, each person can only infect others during the
first cycle after being infected. It is not possible for a person to infect others during multiple cycles for longer lasting diseases. Further, it is not possible for a person
to become infected a second time because there are no loops within the tree.

\section{SIR Model}
The SIR Model is a more advanced model that allows modeling of most social network structures
by generalizing the contact structure. Easley and Kleinberg \cite{networks} define three 
stages each node can have:
\begin{itemize}
    \item \textbf{S}usceptible: Node is not yet infected but susceptible to infection from its neighbors
    \item \textbf{I}nfectious: Node has caught the disease and has a probability to infect its neighbors
    \item \textbf{R}emoved: The node went through the full infection period and is removed from consideration for future cycles
\end{itemize}
The SIR Model uses a directed graph do indicate which nodes are neighbors and thus susceptible 
to infecting each other. The resulting graph does not have to be anti-symmetric it may also contain
undirected edges. 

In addition to the network the SIR Model uses two additional quantities to control the 
epidemic: $p$ the probability an infected node infects a susceptible node and $t_I$ the length
of the infection period.

Initially some nodes are in the $I$ state while all others are in the $S$ state. After
each cycle all neighbors of the nodes in the $I$ state are infected with a probability of $p$.
After $t_I$ steps a node is removed from the $I$ state and can no longer infect others or be
infected by others.

\subsection{Reproductive Number}
\label{sub:r0}
For these more complex networks calculating the reproductive number $R_0$ is not as trivial
as for a tree based network. The book contains an example for a network shown in figure \ref{fig:narrow_network}
in which even highly contagious diseases will die out.

\begin{figure}
    \centering
    \includegraphics[width=0.5\linewidth]{images/narrow_network.png}
    \caption{A network where the disease must pass through a narrow structure of nodes. (source: \cite{networks})}
    \label{fig:narrow_network}
\end{figure}

Let $d$ be a disease with a high contagiousness of $p=0.8$ and an infection period of $t_I=1$. 
Using the assumptions made for calculating $R_0$ in the tree network, this would result 
in $R_0 = 2 \cdot 0.8 = 1.6$, indicating that the disease is relatively likely 
never to die out. However, due to the structure of the network, the disease is going to 
die out relatively quickly. The probability of the disease not spreading during a cycle is
$0.2^4=0.0016$, so on average the diseases will die out after $\frac{1}{0.0016}=625$ cycles.
This shows that in more complex networks, the structure of the network also plays a big role
in the progression of the epidemic. It is not possible to calculate $R_0$ from the
characteristics of the disease alone, the network structure must also be taken into account.
Knowing this, it is impossible to accurately calculate $R_0$ in a highly complex network.
It is however possible to observe the value of $R_0$ during the epidemic for previous cycles.
Let $R_0^k$ be the $R_0$ value for cycle $k$ and $I_k$ the number of infected nodes in cycle $k$.
Then $R_0^k = \frac{I_k}{I_{k-1}}$, with this the evolution of the $R_0$ value can be 
observed, making it possible to estimate $R_0$ for future cycles and helping to understand
whether the disease is currently dying out ($R_0<0$) or not ($R_0>1$). A simulation can also
help to understand how the $R_0$ value of diseases will behave for a disease with certain 
quantities in a complex network.

\subsection{Limitations}
There are still some limitations to the SIR Model. Each person can only catch the disease at
most once and the model does not allow for simulating multiple diseases at the same time.
However, the model allows most social networks to be represented, making it relatively easy
to extend this network to include more complex diseases, e.g. with different levels of contagiousness
depending on the time a node has been infected or a non-constant infection period.

\section{SIS Model}
The SIS Model is an extension to the SIR Model that also allows nodes to be reinfected multiple
times. The \textbf{R}emoved state is exchanged with the \textbf{S}usceptible state, so nodes
that are removed from the \textbf{I}nfected state are placed back in the \textbf{S}usceptible state.
With this change, it is possible for diseases to survive an infinite amount of time in an SIS Model, as opposed to an SIR Model, where each simulation ends after a finite number of steps after the 
disease has burned through all nodes. The only way for the simulation in a SIS Model to end is
if all infected nodes fail to transmit the disease to any of their neighbors $t_I$ times.

\section{Custom Model}
\label{sec:custom_model}
\subsection{Parameters}
The model used by the developed app is a further extension of the SIR and SIS Models. It 
combines the SIR and SIS Models by moving nodes that finish the infection period either
into the \textbf{R}emoved or \textbf{S}usceptible state depending on a quantity $f$. Each
disease has a mortality rate $f$ that determines whether a node is moved into the 
\textbf{R}emoved state and considered dead or moved back into the \textbf{S}usceptible
state if it survived the infection. The \textbf{S}usceptible state is split into two
substates: nodes that have never been infected before and nodes that have been put back into it after being infected at least once. This allows for different infectiousness values $p_I$ for the
first infection of a node and $p_r$ for reinfections. 

The $t_I$ parameter will be split into two new parameters: $t_{min}$ the minimum time an infection lasts before the node can be cured and $t_\rho$ the probability a node is cured from its infection
after the minimum time $t_{min}$ has elapsed. This results in four states so far:

\begin{itemize}
    \item Healthy: Susceptible nodes that were never infected before. They are at risk of being
    infected by their neighbors with a probability of $p_I$
    \item Infected: Nodes that are currently infected. The infection duration is least $t_{min}$ cycles and
    the exact duration is determined by the probability $t_\rho$. After the infection ended the
    node is moved into the cured state with probability $1-f$ or into the deceased state with
    probability $f$.
    \item Cured: Nodes that were infected but survived. They are at risk of being
    infected by their neighbors with a probability of $p_r$ 
    \item Deceased: Nodes that died from the infection. They are not considered in future cycles
    and can not infect others anymore.
\end{itemize}

The infectiousness of a disease usually changes over the course of the infection. In most
cases, an infected person is most likely to infect others during the first few days of contracting
the disease. This should also be represented in the custom model. Thus, the values of $p_I$ and $p_r$ need to change with the time $t_c$ for which a node has been infected. These time-dependent probabilities
will be called $p_I^t$ and $p_r^t$. A healthy node now has a probability of $p_I^t$ to
be infected by another node that has been infected since $t$ cycles. Analogously, a cured node 
now has a probability of $p_r^t$.

Another part of epidemics that is not represented in the SIR or SIS Model are vaccines.
During the course of an epidemic, vaccines can be used to reduce the fatality rate
for infected persons and reduce the likelihood that vaccinated persons will contract the disease. To accommodate this in the custom model, two new node states are added:
\begin{itemize}
    \item Vaccinated: Nodes that are vaccinated and now have a probability of getting infected
    of $p_v^t$ and a fatality rate of $f_v$.
    \item Unvaccinated: Nodes that are not vaccinated and use the above mentioned probabilities
    $p_I^t$/$p_r^t$ and $f$.
\end{itemize}
These two new states are not exclusive with the previous four states. Each node simultaneously
has one of the two states, vaccinated or unvaccinated, and one of the previous four states 
healthy, cured, infected or deceased.

\subsection{Network model}
The network model is very similar to the one used by the SIR and SIS Model. However it does
not allow for directed connections as there are very limited uses for those connections. 
Almost every human contact is bidirectional if for example one person gets close enough to
another person to contract an air transmitted disease this transmission can always happen
in both directions.

The network model used in this custom model organizes the nodes of one social circle in groups
to more clearly organize the network for more visual clarity. Each group can be 
considered a small-world contact network were the nodes in each group are a localized part
of the network that is highly connected with fewer connections to other groups.

\section{Characteristics of epidemics}
\subsection{Oscillating diseases}
As Easley and Kleinberg \cite{networks} explain, diseases with certain characteristics
can cause an oscillating number of infections. Consider a network with several highly
connected groups that have have few connections between the groups. If a disease
with a very high infection probability $p\geq0.9$, a period of immunity $i > 0$ and a
fatality rate close to zero breaks out in such a network, the umber of infections will oscillate.
An example of how the number of infections might evolve over time can be seen in figure 
\ref{fig:oscillation}.
\begin{figure}[!ht]
    \centering
    \includegraphics[width=0.75\linewidth]{images/oscillation.png}
    \caption{Amount of infections over time in a network of highly connected 
    groups. The disease has a high infection rate and low fatality. (source: \cite{networks})}
    \label{fig:oscillation}
\end{figure}

This happens because due to the high infectiousness of the disease, almost all nodes in 
a group that has at least one infected node will be infected within a few cycles. Since
almost all nodes are infected after only a few cycles, there are only a few new infections 
in the next cycles due to the sparse set of available healthy nodes. If the initial
wave started at cycle $c$, then at cycle $c' = c + t_i + i$ the nodes of the first wave 
are susceptible to infection again. This drastically increases the number of possible targets for new infections, which in turn increases the number of new infections again. This results in 
the oscillation of the number of new infections.

Such a scenario can be modeled with the created tool. The network consists of 5 groups with 2,000 nodes each. The groups have a lot of intra-group connections (5 per node) and only a small number of connections to the
other groups. Each group is connected to two other groups with 1 edge per node and it is ensured that
all groups have a path to all other groups.

The disease that will be simulated in this network has an infection rate of 0.2, a fatality
rate of 0, a infection duration of 5 cycles and a immunity period of 3 cycles. 
\clearpage For simplicity the reinfection rate is the same as the initial infection rate and no vaccinations are used.

\begin{figure*}
    \centering
    \begin{subfigure}[b]{0.475\textwidth}
        \centering
        \includegraphics[width=\textwidth]{images/oscillation0.png}
        \caption[Network2]%
        {{\small Network after 0 Steps \newline}}    
        \label{fig:mean and std of net14}
    \end{subfigure}
    \hfill
    \begin{subfigure}[b]{0.475\textwidth}  
        \centering 
        \includegraphics[width=\textwidth]{images/oscillation12.png}
        \caption[]%
        {{\small Network after 12 Steps, infections are at a maximum}}    
        \label{fig:mean and std of net24}
    \end{subfigure}
    \vskip\baselineskip
    \begin{subfigure}[b]{0.475\textwidth}   
        \centering 
        \includegraphics[width=\textwidth]{images/oscillation18.png}
        \caption[]%
        {{\small Network after 18 Steps, infection are at a minimum}}    
        \label{fig:mean and std of net34}
    \end{subfigure}
    \hfill
    \begin{subfigure}[b]{0.475\textwidth}   
        \centering 
        \includegraphics[width=\textwidth]{images/oscillation24.png}
        \caption[]%
        {{\small Network after 24 Steps, infection are at a maximum again}}    
        \label{fig:mean and std of net44}
    \end{subfigure}
    \caption[ State of the network showing the oscillation of infected nodes ]
    {\small State of the network showing the oscillation of infected nodes. Red nodes are currently infected, dark green ones have never been infected and 
    light green ones were previously infected but have recovered and have immunity.} 
    \label{fig:osc_network}
\end{figure*}
\clearpage

Initially, 2 random nodes are infected with the disease. Figure \ref{fig:osc_network}
shows the state of the network after 0, 12, 18 and 24 cycles.
The number of infections over time can be seen in figure \ref{fig:oscillation_p02}
which clearly shows the oscillatory nature of this disease. Over time, the amplitude of the
waves decreases, as the number of people infected at the same time and thus getting cured
at the same time decreases, so the infections are no longer synchronized and each cycle has the same number of new nodes available to be infected. This happens because the
infection rate was too low. The disease was not able to explosively infect all new nodes
as soon as they became available, so some nodes remained uninfected for 2-3 cycles,
shifting their cure time and breaking the oscillation. Increasing the infection rate
to $p = 0.3$ prevents this, as the disease now spreads fast enough to infect every
available node immediately. The result of this is shown in figure \ref{fig:oscillation_p03}.

\begin{figure}[!ht]
    \centering
    \includegraphics[width=0.75\linewidth]{images/oscillation_infections.png}
    \caption{Amount of infections over time showing the oscillating nature with $p = 0.2$}
    \label{fig:oscillation_p02}
\end{figure}

\begin{figure}[!ht]
    \centering
    \includegraphics[width=0.75\linewidth]{images/oscillation_infections3.png}
    \caption{Amount of infections over time showing the oscillating nature with $p = 0.3$}
    \label{fig:oscillation_p03}
\end{figure}
\clearpage

Now the same network is used but the infection rate of the disease is decreased to 0.1.
Because the disease is now not spreading as explosively as before it always has enough targets
to infect until the previously infected nodes become cured again. Thus the amount of new 
infections is more consistent and does not oscillate as seen in figure \ref{fig:no_oscillation}.

\begin{figure}[!ht]
    \centering
    \includegraphics[width=0.75\linewidth]{images/no_oscillation.png}
    \caption{Amount of infections over time showing that no oscillations occur if with $p = 0.1$}
    \label{fig:no_oscillation}
\end{figure}

Another way to break the oscillation is to keep $p=0.3$ but decrease the duration of the infection to 2 cycles 
and remove the immunity period. Now the infected nodes become cured so fast that there are
enough new nodes to infect for every cycle, again resulting in a relatively consistent
amount of new infections as indicated in figure \ref{fig:no_oscillation2}

\begin{figure}[!ht]
    \centering
    \includegraphics[width=0.75\linewidth]{images/no_oscillation2.png}
    \caption{Amount of infections over time showing that no oscillations occur even with $p = 0.3$ if the duration is too short}
    \label{fig:no_oscillation2}
\end{figure}
\clearpage

\subsection{Epidemics without diseases}
Diseases are not the only thing that is capable of spreading through social networks. As
previously mentioned computer viruses spread in a similar pattern. However, there are other
completely different things that can spread in social networks in a similar way.
Information, for example, can be modeled using a very similar approach to the one proposed
for diseases. Information can be passed from person to person whenever they talk to each other,
be it over a long distance through things like text messages or by meeting in person.

Other things that can spread through social networks include traditions or certain habits.
Looking at the world, people within each country are much more likely to interact with other people from the same country than with people from other countries. This means each
country can be modeled as a highly connected group of people with few connections to other
countries. This model can be used to explain how each country has its own traditions that
differ from other countries the further away they are. Traditions spread easily within 
a country, but the further away a country is, the less likely the tradition is to spread to that country.
Because there are few connections to other countries, foreign traditions rarely spread to other countries and if they do, they are drowned out by the explosively spreading regional traditions of the other country.

These models can also be used to model genetic inheritance. As suggested by
Easley and Kleinberg \cite{networks} this can be done by connecting parents to their offspring.
Simulations can then be run to show the spreading of various genetic factors through
the ancestry trees. This type of simulation itself offers many areas that can be expanded
to create various simulations relevant to understanding human history and origins. 
It can also be used to indicate and visualize the existence of common ancestors and 
can give an idea of how many generations it takes to reach them. One such model that deals
with genetics and common ancestors is the Wright-Fisher Model. Haller and Messer \cite{genetics}
go into more detail on how this model and the genetic simulation work. Genetic simulation 
and the Wright-Fisher Model are a large research area in themselves, there are several other
works that discuss these topics, explain the mathematics behind those models, 
possible uses etc. \cite{genetics2} \cite{genetics3} \cite{genetics4}.


These different applications shows how the use of epidemics and social networks goes way beyond modeling diseases and can help understand complex relations in various different areas of research.

\chapter{Implementation}
\label{cha:implementation}
An app that allows for the simulation of epidemics is created. First the required functionalities
of the app need to be determined. It must be able to simulate different epidemics cases for
which different social networks and diseases are required.

\section{Network Editor}
The app needs an editor that allows for the creation of different social networks. Since
social networks can consist of a huge amount of people this editor needs to allow the user
to quickly create networks with a high amount of nodes and connections. For this there
will be three settings for a group of nodes:

\begin{itemize}
    \item \textbf{Size}: The amount of nodes in the group
    \item \textbf{Intra group connections}: The amount of edges each node has to other nodes
    of the same group
    \item \textbf{Intra group connections delta}: The variation for the amount of edges each
    node has ot other nodes of the same group. Eg. a connection amount of 5 with a delta of 3
    would result in each node having between 2 and 8 connections.
\end{itemize}

A network then consists of any amount of groups each with different settings. To allow
for connections between these groups there are similar settings availabe for each pair of
groups:

\begin{itemize}
    \item \textbf{Inter group connections}: The amount of edges each node has to other nodes
    of the other group
    \item \textbf{Inter group connections delta}: The variation for the amount of edges each
    node has ot other nodes of the other group. Eg. a connection amount of 5 with a delta of 3
    would result in each node having between 2 and 8 connections.
\end{itemize}

Using these properties the user is able to quickly generate big networks to model most
social situations. To visualize the structure of the network there will also be a 3D view
of the current network which can be updated after each change. Chapters \ref{cha:network_generation}
and \ref{cha:network_display} explain the implementation of this feature in more detail.

\section{Disease Editor}
To create the diseases that spread in the network the app has a tab were the required properties
for each diseases can be set. These include the properties discussed for the custom model in
section \ref{sec:custom_model} as well as some other properties required for visualizing the network:
\begin{itemize}
    \item \textbf{Name}: Name of the disease, will be shown in the legend when displaying the network
    \item \textbf{Color}: Color nodes infected with this disease will have
    \item \textbf{Fatality rate}: $f$, the chance a node is moved into the deceased state after the infection period is over
    \item \textbf{Vaccinated fatality rate}: Fatality rate for vaccinated nodes
    \item \textbf{Infection rate}: $p_I$, the chance an unvaccinated node gets infected by a neighbor
    \item \textbf{Reinfection rate}: $p_r$, the chance a previously infected node gets infected again by a neighbor
    \item \textbf{Vaccinated infection rate}: $p_v$, the chance a vaccinated node gets infected by a neighbor
    \item \textbf{Minimum duration}: $t_{min}$, the minimum cycles an infection lasts
    \item \textbf{Cure chance}: $t_\rho$, chance a node gets cured (or dies) each cycle after it has been
    infected for $t_{min}$ cycles
    \item \textbf{Immunity period}: Amount of cycles a node is immune after being cured
    \item \textbf{Infectiousness factor}: Decrease of infection rates with each cycle the node has been infected.
    Eg. with a initial $p_I = 0.5$ and a factor $I = 0.9$ the infection rate after $x$ cycles is $p = p_I \cdot I^x$.
    \item \textbf{Initial infections}: Amount of nodes infected with this disease in cycle 0, the start of the simulation
\end{itemize}

\section{Simulation}
To simulate diseases the app contains two different tabs. One where the network is displayed
in 3D and the current state of each node is indicated by colors. The other tab only contains
text for each group showing how many nodes are infected, deceased, etc. per group. The simulation
without visually seeing the graph allows for faster simulation of a high amount of steps and for
networks with a high amount of nodes (over 100,000, depends on computing power of the system)
building the graph might take a long time making the simulation with the visual representation
almost unusable.

Both simulations contain buttons to advance one step, automatically advance steps over time,
reset the simulation and save the statistics collected during the simulation. The simulation
which displays the networks also contains buttons to alter the display of the network, eg. to hide
certain node groups or edges to allow viewing only the areas of the network the user wants to observe.
Chapter \ref{cha:visual_simulation} explains the implementation of this feature in more detail.

\section{Statistics}
The last tab of the app can be used to view the statistics collected during simulations.
It displays the individual stats in a coordinate system and contains functions to split
the stats by different parameters, eg. show only infections with a certain disease or all
diseases or of only a certain group. The graph can be viewed as the individual value of each
step as well as the cumulative value up to each step.

\section{Frameworks}
The app is constructed using python. For the UI QT \cite{qt} is used, this is further explained
in chapter %TODO
For the display of the graphs the plotly \cite{plotly} is used along with the dash implementation
it provides for creating webpages. The graphs will then be displayed on the dash webpage
which is embedded in the QT app. The implementation of the website is explained in chapter %TODO.
For saving a project the JSON file format is used. The saved project files contain all 
necessary information, like the groups of the network, diseases and stats of previous simulations
to allow closing the app after saving without losing any progress.


\chapter{Network Generation}
\label{cha:network_generation}
%TODO add UI stuff
The created tool provides a UI for generating networks to simulate epidemics. To create these networks different groups of people can be defined. Each group consists of $n$ nodes, the members, each having a certain number of edges to other nodes of the same group. The amount of edges per node is specified using an average value $\mu$ and a delta $\delta$. Each node will then have an edge amount between $\mu - \delta$ and $\mu + \delta$.

Let $g1$ and $g2$ be two different groups. Then in addition to the above mentioned edges inside a group, it is also possible to specify the number of edges between the nodes of the two groups with an average and delta value, in the same way as described above.

To create a network graph that meets all of these requirements multiple algorithms are needed.

\section{Creating edges within a group}
\label{sec:creating_edges_in_group}
Let $g$ be a group of $n$ nodes which each needs between $\mu - \delta$ and $\mu + \delta$ edges. Then the first step is to generate a sequence of $n$ integer numbers inside this range. This sequence represents the degree each node should have at the end.

\subsection{Randomly adding new edges}
\begin{algorithm}
\caption{Adding random edges}
\label{alg:random_edges}
\begin{algorithmic}
\State $nodes \gets $nodes with less than required degree
\While{$nodes$ is not empty}
    \State $n1, n2 \gets$ two unconnected nodes
    \State connect $n1$ and $n2$
    \If {$n1$ has required degree}
        \State remove $n1$ from $nodes$
    \EndIf
    \If {$n2$ has required degree}
        \State remove $n2$ from $nodes$
    \EndIf
\EndWhile
\end{algorithmic}
\end{algorithm}

This algorithm was the first iteration for creating the networks, it has two issues. 
\newline

Firstly not every sequence of degrees is graphic, e.g. not every sequence of degrees has a corresponding simple graph. A simple graph is a graph consisting only of undirected edges and no loops.

Let for example $s$ be a sequence of random integers with sum $m$. If $m$ is not even $s$ is definitely not graphic because in a simple graph without loops each edge increases the total degree of all nodes by two. Thus it is impossible to have a total degree of all nodes that is not even. This is not checked in the above algorithm \ref{alg:random_edges} which would cause the $nodes$ list to only have one node at the end making it impossible to select two unconnected nodes from it.
\newline

The second issue is that by adding edges randomly it is possible to end up in a situation, where all remaining edges are already connected but have still not reached the required degree.

\begin{figure}
    \centering
    \includegraphics[width=0.5\linewidth]{images/impossible_graph.png}
    \caption{Graph created by random algorithm \ref{alg:random_edges} with $\mu=2$ and $\delta=0$}
    \label{fig:impossible_graph}
\end{figure}

Figure \ref{fig:impossible_graph} shows one such situation. In this graph each node needs to have a degree of exactly two. In theory it is possible to create such a graph, but the random algorithm \ref{alg:random_edges} may end up in the situation shown in the figure. Here the only two nodes remaining in the $nodes$ list are the two on the right which are already connected. From this state it is impossible to create a graph that satisfies the degree requirements.
\newline

The second iteration of the algorithm uses a more methodical approach to adding the edges to solve the above issues.

\subsection{Erdos-Gallai Theorem}
To solve the first issue of degree sequences that are not graphic, the Erdos-Gallai Theorem is used. It provides one of two known approaches to solving the graph realization problem, i.e. it gives a necessary and sufficient condition for a finite sequence of natural numbers to be the degree sequence of a simple graph.

\begin{equation}
\sum_{i=1}^{k} d_i \leq k(k-1) + \sum_{i=k+1}^{n} \min(k, d_i)
\end{equation}
for all integers \(k\) with \(1 \leq k \leq n\), where \(d_1 \geq d_2 \geq \ldots \geq d_n\) are non-negative integers.

A sequence of degrees is graphic if and only if \(d_1 + d_2 + \ldots + d_n\) is even and the above equation holds for every $k$. The inequlity ensures that the sum of the first $k$ terms of the degree sequence does not exceed the theoretical maximum number of edges $(k(k-1))$ plus the sum of the remaining edges $\sum_{i=k+1}^{n} \min(k, d_i)$ for the remaining vertices.

For this tool every degree sequence needs to be graphic because it should always be possible to create a graph for the inputs the user made. This means if the Erdos-Gallai Theorem shows that a sequence is not graphic, it needs to be changed until it is. Achieving this is done by simply decrementing a random number of the degree sequence by one.

If the Erdos-Gallai Theorem failed because the sum was not even, the sum will be even after decrementing once. By decrementing random numbers of the sequence only the left side of the inequality is getting smaller, because $k$ is never changed. This means after decrementing enough times the Theorem will always be fulfilled. Thus every sequence of degrees can be changed to be graphic.

The python implementation of this algorithm can be seen in \ref{lst:erdos_gallai}.

\begin{lstlisting}[language=python, caption={Erdos-Gallai Theorem in Python}, label={lst:erdos_gallai}]
def _erdos_gallai(self) -> bool:
    if sum(self.deg_seq) % 2 != 0:
        return False
    n = self.size
    for k in range(1, n + 1):
        if sum(self.deg_seq[:k]) > k * (k - 1) + sum(min(k, d) for d in self.deg_seq[k + 1 :]):
            return False
    return True
\end{lstlisting}

\subsection{Havel-Hakimi algorithm}
\label{sub:havel_hakimi}
After creating a graphic degree sequence using the Erdos-Gallai Theorem, with all degrees within the $\mu - \delta$ and $\mu + \delta$ range, this degree sequence needs to be converted to a network graph.

\begin{algorithm}
\caption{Havel-Hakimi algorithm}
\label{alg:havel_hakimi}
\begin{algorithmic}
\State $deg\_seq \gets $ graphic sequence of degrees
\State $nodes \gets $ list of nodes, with same length as $deg\_seq$
\While{$sum(deg\_seq) > 0$}
    \State $n \gets nodes.pop(0)$
    \State $d \gets deq\_seq.pop(0)$
    \State $targets \gets $ n nodes with highest degree from $nodes$
    \For{$t$ in $targets$}
        \State connect $t$ and $n$
        \State $deg\_seq[t] \gets deg\_seq[t] - 1$
    \EndFor
\EndWhile
\end{algorithmic}
\end{algorithm}

Using the Havel-Hakimi algorithm \ref{alg:havel_hakimi} a simple graph can be constructed for every graphic sequence of degrees. This algorithm always finds a correct solution %TODO add ref to proof

\subsubsection{Python implementation}
The Havel-Hakimi algorithm is implemented inside the HavelHakimi class. This class has properties for the sequence of degrees \texttt{deg\_seq}, list of node ids \texttt{node\_id\_seq} and a dictionary which nodes are connected \texttt{edges} (each key node id has an undirected edge to all its value node ids).

First the \texttt{node\_id\_seq} is sorted to be non increasing and the \texttt{node\_id\_seq} is shuffled to assign random degrees to each node. Then the Havel-Hakimi algorithm is used:

\begin{lstlisting}[language=python, caption={Havel Hakimi Algorithm in Python}, label={lst:havel_hakimi}]
def _connect_nodes(self):
    while sum(self.deg_seq) > 0:
        node = self.node_id_seq.pop(0)
        deg = self.deg_seq.pop(0)
        targets = self._get_highest_n_nodes(deg)
        for target in targets:
            self.deg_seq[self.node_id_seq.index(target)] -= 1
        self.edges[node] = targets
        self._sort_sequence()
\end{lstlisting}

The function in listing \ref{lst:havel_hakimi} implements the above algorithm \ref{alg:havel_hakimi}. After each iteration the degree sequence is sorted so it is nonincreasing again. When sorting the \texttt{deg\_seq} it is important that the \texttt{node\_id\_seq} is altered in the same way so each node id still corresponds to the same degree, this function is shown in \ref{lst:sorting}.

\begin{lstlisting}[language=python, caption={Sorting degrees and node ids}, label={lst:sorting}]
def _sort_sequence(self):
    self.node_id_seq = [x for _, x in sorted(zip(self.deg_seq, self.node_id_seq), reverse=True)]
    self.deg_seq.sort(reverse=True)
\end{lstlisting}

\section{Creating edges between groups}
Creating the edges between two groups randomly results in similar problems as described in section \ref{sec:creating_edges_in_group}. So a modified version of the Havel-Hakimi algorithm and Erdos-Gallai Theorem is used.

\subsection{Creating the degree sequence}
Let $g1$ and $g2$ be two disjunct groups of sizes $n$ and $m$. If $n \neq m$ it might not be possible that each node has a degree between $\mu - \delta$ and $\mu + \delta$. Let $n = 5$ and $m = 10$ with $\mu = 5$, $\delta = 0$. If every node in $g2$ has a degree of $5$ all nodes in $g1$ would have a degree of $10$. If every node in $g1$ has a degree of $5$ the average degree of the nodes in $g2$ would only be $2.5$.

Thus if $n \neq m$ one group might have lower degrees than the minimum or higher degrees than the maximum for certain $\mu$ and $\delta$. In this work the solution where the bigger group might have a lower degree will be used.
\newline

First the degree sequence for the smaller group $g1$ will be created. The sequence consists of $n$ integer numbers between $\mu - \delta$ and $\mu + \delta$. The degree sequence of the $g2$ must have the same sum as the sequence of $g1$ or else the sequences are not graphic because each added edge decreases the sum of the degree sequences of $g1$ and $g2$ by one and the Havel-Hakimi algorithm finishes only when both sequences reach $0$.
\newline

Let $s1$ be the sum of the degree sequence for $g1$. An algorithm is needed to create a sequence of $m$ integer numbers with sum $s1$ while the numbers should be as close to or inside the desired range $\mu - \delta$ to $\mu + \delta$ as possible.

\begin{algorithm}
\caption{Sequence generation}
\label{alg:seq_with_sum}
\begin{algorithmic}
\Require {$(g1, m, g1\_deg\_seq, \mu, \delta$)}
\Ensure {$g2\_deg\_seq$}
\State $s1 \gets sum(g1\_deg\_seq)$
\State $seq \gets $ sequence of $m$ random integers between $\mu - \delta$ and $\mu + \delta$
\While{$sum(seq) != s1$}
    \If{$sum(seq) > s1$}
        \State $indices \gets []$
        \For {$deg$ in $seq$}
            \If{$deg > \mu - \delta$}
                \State $indices$ push $deg$
            \EndIf
        \EndFor
        \If{length of $indices$ == 0}
            \For {$deg$ in $seq$}
            \If{$deg > 0$}
                \State $indices$ push $deg$
            \EndIf
            \EndFor
        \EndIf
        \State $i \gets $ random entry of $indices$
        \State $seq[i] \gets seq[i] - 1$
    \Else
        \State $indices \gets []$
        \For {$deg$ in $seq$}
            \If{$deg < \mu + \delta$}
                \State $indices$ push $deg$
            \EndIf
        \EndFor
        \State $i \gets $ random entry of $indices$
        \State $seq[i] \gets seq[i] + 1$
    \EndIf
\EndWhile
\end{algorithmic}
\end{algorithm}

The algorithm \ref{alg:seq_with_sum} first creates a random sequence of integers within the $\mu - \delta$ to $\mu + \delta$ range. If the sum of this sequence is too big it decrements random entries until all degrees are equal to $\mu - \delta$. If this is still too big random entries are decremented further until the required sum is reached. If the initial sum of the sequence is too small random entries are incremented until the required sum is reached. It is always possible to each the sum without a single element of the sequence being bigger than $\mu + \delta$ because the length of this sequence will always be smaller than or equal to the sequence of the first group. Thus the maximum sum of the $g1_seq$ is $n \cdot (\mu + \delta)$ and the maximum sum of $g2_seq$ is $m \cdot (\mu + \delta)$. Because $n \leq m$ the following always holds: $n \cdot (\mu + \delta) \leq m \cdot (\mu + \delta)$. So the sequence of length $m$ that is generated using the algorithm \ref{alg:seq_with_sum} will never be bigger than the maximum of degrees within the $\mu - \delta$ to $\mu + \delta$ range.

The python implementation of this algorithm is shown in listing \ref{lst:seq_with_sum}.

\begin{lstlisting}[language=python, caption={Algorithm for creating a sequence with specific sum}, label={lst:seq_with_sum}]
def _create_sequence_with_sum(self, size: int, _sum: int):
    seq: List[int] = np.random.randint(self.min_deg, self.max_deg + 1, size=(size)).tolist()
    while sum(seq) != _sum:
        if sum(seq) > _sum:
            if len(choices := [x for x in seq if x > self.min_deg]) == 0:
                choices = [x for x in seq if x > 0]
            seq[seq.index(random.choice(choices))] -= 1
        else:
            seq[seq.index(random.choice([x for x in seq if x < self.max_deg]))] += 1
    return seq
\end{lstlisting}

Because the second degree sequence is always constructed so that the two sequences together are graphic, the Erdos-Gallai Theorem is theoretically no needed here. It is still used to check the result in case there are any implementation errors that produce a non graphic sequence, though it should always return true if the implementation has no issues. The modified implementation of the Erdos-Gallai Theorem can be seen in listing \ref{lst:modified_erdos_gallai}. This implementation tests the inequality for two sets of values: $n$, the size of $g1$, with $g2_deg_seq$ and $m$ the size of $g2$ with $g1_deg_seq$. This combination of values is used because the nodes of $g2$ have to be connected to the $n$ nodes of $g1$ and the other way around. If the second degree sequence is constructed using the above algorithm \ref{alg:seq_with_sum} this function will always return true.

\begin{lstlisting}[language=python, caption={Modified Erdos-Gallai Theorem for connecting to disjunct groups}, label={lst:modified_erdos_gallai}]
def _erdos_gallai(self) -> bool:
    for n, seq in zip([self.size1, self.size2], [self.deg_seq2, self.deg_seq1]):
        for k in range(1, n + 1):
            if sum(seq[:k]) > k * (k - 1) + sum(min(k, d) for d in seq[k + 1 :]):
                return False
    return True
\end{lstlisting}



\subsection{Modified Havel-Hakimi algorithm}
The Havel-Hakimi algorithm from section \ref{sub:havel_hakimi} is modified to use two disjunct groups of nodes it connects. The implementation of the modified algorithm can be seen in listing \ref{lst:modified_havel_hakimi}. Before this algorithm is used the two degree sequence are constructed and sorted to be non increasing. The two node id sequence are also created and shuffled.

\begin{lstlisting}[language=python, caption={Modified Havel-Hakimi algorithm}, label={lst:modified_havel_hakimi}]
def _connect_nodes(self):
    while sum(self.deg_seq1) + sum(self.deg_seq2) != 0:
        if self.size1 <= self.size2:
            node = self.node_id_seq1.pop(0)
            deg = self.deg_seq1.pop(0)
            targets = self._get_highest_n_nodes(deg, self.deg_seq2, self.node_id_seq2)
            for target in targets:
                self.deg_seq2[self.node_id_seq2.index(target)] -= 1
        else:
            node = self.node_id_seq2.pop(0)
            deg = self.deg_seq2.pop(0)
            targets = self._get_highest_n_nodes(deg, self.deg_seq1, self.node_id_seq1)
            for target in targets:
                self.deg_seq1[self.node_id_seq1.index(target)] -= 1
        self.edges[node] = targets
        self._sort_sequence()
\end{lstlisting}

The algorithm creates the edges starting from the smaller group. Let $d$ be the highest degree from the smaller group. Then a node from the smaller group with degree $d$ is selected for \texttt{node}. After that the $d$ nodes with the highest degree are selected from the list of nodes of the \textbf{other} (bigger) group. This is the main difference to the original Havel-Hakimi algorithm which selects the nodes from the same group. Then the \texttt{node} from the smaller group is connected to each selected node from the bigger group and the degrees for all nodes \texttt{node} connects to are decremented.

\chapter{Network Display}
\label{cha:network_display}
Originally, it was intended to display the networks in 2D, with a 3D representation being considered a nice to have feature. During testing we found that a 2D representation does not provide a clear view of the network. Even for relatively small networks (\textasciitilde 200 nodes) the 2D representation was very confusing. For this reason, the 2D representation was scrapped and only the 3D one was implemented. In 3D, even networks with thousands of nodes provide a clear view of the different groups and nodes.
\newline

The Python package plotly \cite{plotly} is used to display the networks in 3D. \texttt{plotly.graph\_objs} provides the \texttt{Scatter3d} function, which is used to draw both the nodes and edges. For the nodes the function takes in three lists of coordinates, one for each axis x, y and z. For the edges it also requires these three lists, but here the first coordinate is the start point for an edge, the second the end point, followed by a None entry and then the next edge coordinates. Currently, the only information that is generated for graphs is the number of nodes in each group and lists of which node ids must be connected with edges. This information must be translated into coordinates for plotly to be able to display the network.

\section{Displaying the nodes}
\label{sub:displayNodes}
All nodes belonging to the same group should be arranged in a sphere. Each sphere thus represents one group and the spheres must have enough space between them so that the groups can be easily distinguished, as shown in Figure \ref{fig:groups}.

\begin{figure}
    \centering
    \includegraphics[width=0.75\linewidth]{images/groups.png}
    \caption{Network containing 4 groups with 2000 nodes per group}
    \label{fig:groups}
\end{figure}

To distribute the spheres in the coordinate system, it is first divided into cubes. Let $n_{max}$ be the number of nodes in the biggest group. To determine the side length $s$ of the cubes, it is necessary to know the radius $r$ of the smallest sphere that can fit $n_{max}$ nodes.

\subsection{Creating a sphere}
The nodes are placed only at coordinates $x,y,z \in \mathbb{N}$. To create a sphere that can fit $n_{max}$ nodes, the radius needs to be known before the exact coordinates are calculated. The problem with this is that the only way to get the exact radius for a sphere that can fit $n_{max}$ nodes that are all at coordinates $x,y,z \in \mathbb{N}$ is to use the algorithm depicted in \ref{alg:exact_radius}.
\begin{algorithm}
\caption{Calculating exact radius}
\label{alg:exact_radius}
\begin{algorithmic}
\Require {$(n_{max})$}
\Ensure {$r$}
\State $r \gets 0$
\State $n \gets 0$
\While{$n < n_{max}$}
    \State $r \gets r+1$
    \State coords $\gets$ calculate coordinates for all nodes in sphere with radius $r$
    \State $n \gets $ length of coords
\EndWhile
\end{algorithmic}
\end{algorithm}
This algorithm is inefficient because it computes all spheres until $r$ is big enough. This can be improved by estimating the number of nodes in a sphere of radius $r$ instead of computing the exact number. The number of nodes is estimated using the formula \ref{eq:sphere_lattice}.
\begin{equation}
\label{eq:sphere_lattice}
    \dfrac{4 \cdot \pi \cdot r^3}{3}
\end{equation}
This estimation incurs an error. \ref{eq:gauss_error} shows the best currently known formula for calculating this error which was discovered and proven by Zheng \cite{gaussSphereProblem}.
\begin{equation}
\label{eq:gauss_error}
    \sum_{\substack{x \in z^3 \\ |x| \leq r}}{P(x)} = O_{\epsilon, P}(r^{v + 84 / 63 + \epsilon})
\end{equation}
The true error bound is suspected to be $O_\epsilon(R^{1 + \epsilon})$ but this is currently not possible to prove. For the context of this work the simpler estimation shown in formula \ref{eq:gauss_error_rough} is sufficient.
\begin{equation}
\label{eq:gauss_error_rough}
    O_{\epsilon}(r^{21/16 + \epsilon})
\end{equation}

Let $r_e$ be the resulting radius required for $n_{max}$ nodes using the estimation. If $n_{max}$ is close to the boundary between two radii, then the resulting radius may be too large or too small due to the introduced error. For this reason, the resulting radius is decremented by one and the exact calculation is then used to find the correct required radius, as shown in algorithm \ref{alg:exact_radius_improved}. This will always find the smallest radius for a sphere containing at least $n_{max}$ points.

\begin{algorithm}
\caption{Calculating exact radius}
\label{alg:exact_radius_improved}
\begin{algorithmic}
\Require {$(n_{max})$}
\Ensure {$r$}
\State $r \gets 0$
\State $n \gets 0$
\While{$n < n_{max}$}
    \State $r \gets r + 1$
    \State $n \gets$ estimate amount of nodes in sphere with radius $r$
\EndWhile
\State $ r \gets r - 1$
\While{len(coords) $ < n_{max}$}
    \State coords $\gets$ calculate exact coords of nodes for radius $r$
\EndWhile
\end{algorithmic}
\end{algorithm}

The formula \ref{eq:sphere_inequality} represents the inequality for points inside or on the surface of a sphere centered at the origin in three-dimensional space, as proven by Gui and Moradifam \cite{sphereInequality}. A point is inside the sphere if this inequality is satisfied.
\begin{equation}
\label{eq:sphere_inequality}
    x^2 + y^2 + z^2 \leq r^2
\end{equation}
This means the bigger one of the values $x,y,z$ becomes, the smaller the other must be to satisfy the inequality. Using this knowledge, it is possible to construct an algorithm (shown in \ref{alg:calc_sphere_points}) that efficiently computes all triples of $x,y,z$ values that satisfy the inequality. The order of $z,x,y$ in the algorithm is important, so a half full sphere creates points that fill the sphere from the bottom to the top.
Starting with $z$ and $y=0$, transforming the formula to $x$ results in $|x| \leq \sqrt{(r^2 - z^2)}$, which corresponds to all possible values of $x$ with respect to $z$ and $r$. This range corresponds to $-a,a$ in the algorithm. Solving for $y$ then results in $|y| \leq \sqrt{(r^2 - z^2 - x^2)}$, which yields all possible values for $y$ with respect to $x,z,r$. This coincides with $-b,b$ in the algorithm.

\begin{algorithm}
\caption{Calculating lattice points}
\label{alg:calc_sphere_points}
\begin{algorithmic}
\Require {$r$}
\Ensure {$coords$}
\State $coords \gets []$
\For {$z := -r$ to $r$}
    \State $a \gets \lfloor\sqrt{(r^2 - z^2)}\rfloor$
    \For {$x := -a$ to $a$}
        \State $b \gets \lfloor\sqrt{(a^2 - x^2)}\rfloor$
        \For {$y := -b$ to $b$}
            \State coords $\gets (x,y,z)$
        \EndFor
    \EndFor
\EndFor
\end{algorithmic}
\end{algorithm}

The Python implementation of this function takes in an amount of points $n$ and returns a list of triples of length $n$ containing the resulting coordinates and the radius $r$ of the sphere.

\section{Arranging the spheres}
After calculating the coordinates and radii for each sphere (one for each group), the biggest radius $r_{max}$ is known. Using this, the side length of the cubes can be calculated. To create some space between the spheres, this side length $s$ is calculated as $s = 2 * r_{max} * 1.25$ to add an empty space of 12.5\% of the sphere diameter on each side.

The cubes are always created within a larger cube, e.g. the number of cubes is $c = x^3$. $x$ is the side length of the bigger cube in cubes (e.g. a side length of $x=3$ cubes results in $c=27$ smaller cubes). In order not to spread out the spheres unnecessarily, the smallest number of cubes should be created. The minimum required number of cubes is the number of spheres/groups. To get the next higher value for $c$, $x$ can be calculated by $x =\lceil\sqrt[3]{num\_groups}\rceil$.

Using this the algorithm \ref{alg:cube_offsets} computes the coordinates of bottom left corner of each cube.

\begin{algorithm}
\caption{Calculating cube offsets}
\label{alg:cube_offsets}
\begin{algorithmic}
\Require {$num\_group, r_{max}$}
\Ensure {$offsets$}
\State $s \gets \lceil 2 * r_{max} * 1.25\rceil$
\State $o \gets \lceil 2 * r_{max} * 0.125\rceil$
\State $c \gets \lceil\sqrt[3]{num\_groups}\rceil$
\For {$x,y,z := 0$ to $c$}
    \State offsets $\gets (x \cdot s + o, y \cdot s + o, z \cdot s + o)$
\EndFor
\end{algorithmic}
\end{algorithm}

For each sphere a random cube is selected and its offsets are added to the coordinates of all points within the sphere, resulting in the final coordinates for all nodes in the network.

\section{Creating the scatter graph}
\label{sub:scatter_graph}
In the final graph it must be possible to toggle the visibility of each group. To achieve this, the coordinates for each group are stored separately and only the active ones are added to the coordinate lists for the \texttt{Scatter3d} call.

\begin{lstlisting}[language=python, caption={Creation of the node trace}, label={lst:node_trace}]
for group in self.network.groups:
    if group.id not in self.hidden_groups:
        x, y, z = zip(*self.group_coords[group.id])
        aXn.extend(x)
        aYn.extend(y)
        aZn.extend(z)
        colors.extend([group.color] * len(x))
trace1 = go.Scatter3d(
    x=aXn,
    y=aYn,
    z=aZn,
    mode="markers",
    name="nodes",
    marker=dict(
        symbol="circle",
        size=6,
        color=colors,
        line=dict(color="rgb(50,50,50)", width=0.5),
    ),
    uirevision="0",
    showlegend=False,
)
\end{lstlisting}

The code in \ref{lst:node_trace} creates the trace that displays all active nodes. Whenever the visibility of a group is toggled, this trace must be rebuilt. It is important to note that the coordinates for the nodes in the x,y,z arrays contain the nodes in ascending order by their id. This means that nodes of group 0 come first starting with node 0, then group 1, and so on. This is important for creating color sequences based on the node states, which is discussed in section \ref{sec:color_sequence}.

\section{Displaying the edges}
The edges are displayed using a second \texttt{Scatter3d} with the \texttt{lines} mode. For this the x,y and z arrays need the format of [xStart, xEnd, None] for each line. This means that a coordinate array for 100 lines would have a length of 300.

The current format in which the edges are represented contains only the node ids that are connected by edges. This representation was created in section \ref{sub:havel_hakimi}. To translate these node ids to the coordinates of the nodes calculated in section \ref{sub:displayNodes}, a map is created when the coordinates are calculated for each node. This map stores the node id as a key and the index of its coordinates, in the coordinate array, as value. This map can be used to calculate the start and end coordinates for all edges of visible groups, which is done by looking up the coordinates of the start and end nodes by using their ids and the map. The python code that realizes this can be seen in listing \ref{lst:edge_creation}.
\begin{lstlisting}[language=python, caption={Creation of the edge coordinate arrays}, label={lst:edge_creation}]
 for edge in edges:
    _from, to = edge[0], edge[1]
    if not (_from in node_id_map and to in node_id_map):
        continue
    from_ind = node_id_map[_from]
    to_ind = node_id_map[to]
    aXe.extend([node_coords_x[from_ind], node_coords_x[to_ind], None])
    aYe.extend([node_coords_y[from_ind], node_coords_y[to_ind], None])
    aZe.extend([node_coords_z[from_ind], node_coords_z[to_ind], None])
\end{lstlisting}

The trace is then created using the \texttt{Scatter3d} call from listing \ref{lst:edge_trace}.
\begin{lstlisting}[language=python, caption={Creation of the edge trace}, label={lst:edge_trace}]
trace2 = go.Scatter3d(
            x=aXe,
            y=aYe,
            z=aZe,
            mode="lines",
            uirevision="0",
            line=dict(color="rgb(125,125,125)", width=1),
            hoverinfo="none",
            showlegend=False,
        )
    \end{lstlisting}

\chapter{Visual Simulation}
\label{cha:visual_simulation}
To simulate epidemics in a network, each node needs to store its state. As explained in %TODO ref
each node can have one of the following states: healthy, cured, infected, vaccinated, deceased. The current state is stored in the node along with the disease it is infected with for the state \texttt{infected}. The amount of cycles the node has been infected and the total amount of times the node has been infected are also stored. Each node also contains all nodes it is connected to.

Using this information the simulation can be done using the algorithm in \ref{alg:simulation}.

\begin{algorithm}
\caption{Simulate epidemics}
\label{alg:simulation}
\begin{algorithmic}
\Require {$nodes$}
\For{node in $nodes$}
    \If {node is not vaccinated}
        \State vaccinate according to vaccination chance of group
    \EndIf
    \If {node is infected}
        \If {infection time > disease duration}
            \State kill or cure node according to rates of disease
        \Else
            \State infection time $\gets$ infection time + 1
            \For{each connected node that is not infected}
                \State infect node according to rate of disease 
                \State (depending on vaccination status and previous infections)
            \EndFor
        \EndIf
    \EndIf
\EndFor
\end{algorithmic}
\end{algorithm}

\section{Displaying the status}
\label{sec:color_sequence}
To display the current state of the simulation the nodes are colored according to their state. Each state has a configurable color with infected having different colors for each disease. Since nodes may have multiple concurrent states (eg. vaccinated and infected) each state has a priority and the color of the highest priority state will be displayed. A lower number means a higher priority.
\begin{enumerate}
    \item Infected/Deceased
    \item Vaccinated
    \item Healthy/Cured
\end{enumerate}

Using this priority list a sequence of colors can be generated from all node states. This sequence is created for the nodes in ascending order of their ids. This is important because the \texttt{Scatter3d} graph requires the colors in the same order as the coordinates which are also in ascending order as mentioned in section \ref{sub:scatter_graph}. The color of the graph can then be updated using the \texttt{update\_trace} method as seen in listing \ref{lst:update_color}

\begin{lstlisting}[language=python, caption={Updating the colors of the graph}, label={lst:update_color}]
fig.update_traces(selector=dict(name="nodes"), marker=dict(color=state_colors))
\end{lstlisting}

\chapter{Experiments}
\label{cha:experiments}
In this chapter some experiments will be conducted. These experiments are modeled
after the scenarios described in the Theory chapter %TODO ref
The goal is to recreate these situation using the developed tool and running a
simulation with the described parameters. These simulations can then be used
to support or refute the assumptions that were made on how the diseases
theoretically should behave in those scenarios.

\section{Importance of $R_0$}
The role of the reproductive number $R_0$ was discussed in section \ref{sub:r0}.
It is the deciding factor for whether a disease is dying out ($R_0 < 1$) 
or able to live indefinitely ($R_0 > 1$). Calculating $R_0$ in complex social
networks is very hard because the structure of the networks has a big impact on $R_0$
in addition to the characteristics of the disease. For these experiments a very
rough estimation of $R_0=e_\mu \cdot p \cdot t_I$ is used as this is sufficient for estimating whether
the disease should die out in the experiment or life for a long time.
$e_\mu$ is the average amount of connection per node and $t_I$ the time it takes
for an infection to be cured, let $N$ be the total
amount of nodes in the network and $E$ the total amount of edges then
$e_\mu=\frac{E}{N}$. $p$ is the probability for a node to get infected
if one of its neighbors has the disease.

\subsection{The network}
\label{sub:exp_network}
The network that is used in these experiments contains 3 groups with 5000 nodes
each. Group 1 has 5 intra group connections for each node, group 2 has 4 and
group 3 has 3 intra group connections. Between each pair of groups there are
2 connections per node. The resulting network can be seen in figure \ref{fig:exp_r0_network}
It contains a total of 15,000 nodes.

\begin{figure}
    \centering
    \includegraphics[width=0.5\linewidth]{images/experiment_r0_network.png}
    \caption{Structure of the network used for the $R_0$ experiments}
    \label{fig:exp_r0_network}
\end{figure}

Let $N_i$ be the amount of nodes in group $i$
then the amount of edges can be calculated using the following equation:
\begin{equation}
    E = \frac{n_1}{2} \cdot 5 + \frac{n_2}{2} \cdot 4 + \frac{n_3}{2} \cdot 3 + \frac{n_1+n_2}{2} \cdot 2 + \frac{n_1+n_3}{2} \cdot 2 + \frac{n_2+n_3}{2} \cdot 2
\end{equation}
Using that equation the it can be calculated that the network contains 60,000 edges.
This leads to an average amount of connections per node of $e\mu=\frac{60,000}{15,000} \cdot 2 = 8$.
But the different groups also need to be viewed in isolation. If the $R_0$ value within
one group of nodes is bigger than one, the disease is likely to never die out in that group
and the nodes of that group will constantly spread the disease to the other groups.
An example of this is shown in the next section.

\subsection{Experiment with $R_0 < 1$}
Since this experiment will use the same network as the one for a $R_0 > 1$ the
facor $e_\mu$ is static and cannot be changed. This makes $p$ the only deciding
factor for $R_0$ it can be calculated using $p = \frac{R_0}{e_\mu \cdot t_I}$ so in this case with $t_I=5$
with $R_0 < 1$, $p < \frac{1}{40}$. For cases with $R_0$ close to 1 there is still
a decent chance that the disease dies out even if $R_0 < 1$. $p=0.024$
is chosen for an estimated $R_0=0.96$ to ensure that the disease dies out in a finite
amount of steps.

The parameters for the disease of this experiment then are:
\begin{itemize}
    \item (vaccinated) fatality rate: 0
    \item (re-, vaccinated) infection rate: 0.024
    \item infection period $t_I$: 5
    \item initial infections: 5
    \item cure chance: 1
    \item immunity period: 0
    \item infectiousness factor: 1
\end{itemize}

Running the simulation shows that the disease is never able to spread to a large
amount of people and dies out after only \~9 steps. This can also be seen in the first graph
of figure \ref{fig:exp_r0_small}
which shows the amount of new infections per cycle. 
Increasing the initial amount of infected nodes to 5,000 shows that with
such a large amount of infections the disease still dies out after only \~20 steps

% is now able to spread to a lot of nodes intially.
% As the disease now has enough infected nodes to avoid dying out at random because of 
% randomness, it can be seen that it actually never dies out, as can be seen in the second graph
% of figure \ref{fig:exp_r0_small}. It constantly has between 50-100 new infections eveen though
% the estimated $R_0$ was only 0.96. This is due to the fact mentioned earlier, that each group
% also needs to be viewed in isolation: Group 1 has 9 connections per node so $R_0^1 = 0.02 * 7 * 5 = 1.08$
% which is bigger than 1. This means the disease has a chance to never die out within group 1 and is constantly
% spreading to the other groups from group 1.

Lowering $p$ to 0.02 results in an $R_0=0.9$ for group 1 and a total estimated $R_0=0.8$,
with these values the disease should die out in a finite amount of steps.
In this case it took \~40 steps for the disease to die out, the right graph of figure \ref{fig:exp_r0_small}
shows the amount of new infections up to that point.

\begin{figure*}
    \centering
    \begin{subfigure}[b]{0.475\textwidth}
        \centering
        \includegraphics[width=\textwidth]{images/exp_r0_small_1.png}
        \caption[Network2]%
        {{\small Infection counts with 5 initial infections}}   
    \end{subfigure}
    \hfill
    \begin{subfigure}[b]{0.475\textwidth}  
        \centering 
        \includegraphics[width=\textwidth]{images/exp_r0_success.png}
        \caption[]%
        {{\small Infection counts with 5000 initial infections, showing that the disease
        still dies out after \~40 steps}}    
    \end{subfigure}
    \caption[ Experiment with an estimated $R_0$ of 0.96 and $p = 0.024$ ]
    {\small Experiment with an estimated $R_0$ of 0.96 and $p = 0.024$} 
    \label{fig:exp_r0_small}
\end{figure*}



This experiment supports the theory that for a $R_0 < 1$ the disease will die out
in a finite amount of steps. 

\subsection{Experiment with $R_0 > 1$}
This network uses the same network as the previous one, so $p$ is again the deciding
factor for $R_0$. $p = \frac{R_0}{e_\mu \cdot t_I}$ so for this case with $R_0 > 1$, $p > \frac{1}{40}$.
The theory created said that it should be sufficient to have a $p$
big enough to ensure $R_0 > 1$ in only one of the groups. Since group 1 has the most connections
$R_0^1 > 1$ if $p > \frac{1}{9\cdot5} = 0.0222$.
Since the closer $R_0$ to 1 the higher the chance the disease randomly dies out $p=0.03$ is chosen
resulting in a $R_0^1 = 1.35$ and $R_0 = 1.2$ where $R_0^1$ is big enough for the disease
to have a decent chance to survive indefinitely.

The properties that were changed from the previous experiment are:
\begin{itemize}
    \item (re-, vaccinated) infection rate: 0.04
    \item initial infections: 50
\end{itemize}

Running the simulation shows that the disease consistenly dies out at around 70 steps.
The left graph in figure \ref{fig:exp_r0_big} shows the infection counts. The expectation
was that the disease has a decent chance to survive but even after 20 tries it never
survived once. This can be explained by the network structure: the nodes in group 1
have 9 connections but 4 of those are "worth less" as the nodes it can spread to in the
other groups are not able to spread the disease to another 9 new nodes but only 7 or 8
depending on the group. This means that when calculating $R_0$ the nodes in group 1 can not
be considered to have 9 connections, somewhere around 8 would be more accurate. This makes
$R_0^1 = R_0 = 1.2$ which is not high enough wor a decent chance of indefinite survival.

Increasing $p$ to 0.036 is enough to make the disease live indefinitely with $R_0 = 1.44$.
After running the simulation for 20,000 cycles the disease is still alive and infecting
new nodes. This supports the theory that with a $R_0 > 1$ there is a probability $> 0$
that the disease never dies out. 


\begin{figure*}
    \centering
    \begin{subfigure}[b]{0.475\textwidth}
        \centering
        \includegraphics[width=\textwidth]{images/exp_big_r0_fail.png}
        \caption[Network2]%
        {{\small Infection counts with 50 initial infections and $p = 0.032$}}   
    \end{subfigure}
    \hfill
    \begin{subfigure}[b]{0.475\textwidth}  
        \centering 
        \includegraphics[width=\textwidth]{images/exp_big_r0_success.png}
        \caption[]%
        {{\small Infection counts with 150 initial infections and $p = 0.036$}}    
    \end{subfigure}
    \caption[Experiment with an estimated $R_0$ of 0.64 and $R_0^1 = 1.12$]
    {\small Experiment with an estimated $R_0$ of 0.64 and $R_0^1 = 1.12$} 
    \label{fig:exp_r0_big}
\end{figure*}

\subsubsection{Changing the network}
To show that the network structure indeed has an impact on $R_0$ this experiment
uses the same disease as the previous one with $p=0.036$ and 150 initial infections.
The network will have only half as many
edges, group 1 has 2 intra group connections for each node, group 2 has 2 and
group 3 has 2 intra group connections. Between each pair of groups there are
1 connections per node. This new network has 30,000 edges resulting in $e_\mu$ = 4.
Thus the new estimated $R_0=2\cdot0.375=0.75$ which is smaller than 1 so they
expectation is that the disease will die out even though it has the same infection 
rate as before.

After 20 cycles the disease has died out which supports the theory that the network structure
and thus $e_\mu$ has an impact on the spreading of diseases and the value $R_0$.
Graph \ref{fig:exp_change_network} shows the amount of infections until the disease died out.

\begin{figure}
    \centering
    \includegraphics[width=0.5\linewidth]{images/exp_changed_network.png}
    \caption{Amount of infections with the same diseases ($p = 0.036$) but a network with half as many connections}
    \label{fig:exp_change_network}
\end{figure}

To support the theory from earlier that each group also needs to be viewed in isolation
when calculating $R_0$ the network is again changed. Group 1 has 8 intra group connections for each node,
group 2 has 1 and group 3 has 1 intra group connections. Between each pair of groups there are
2 connections per node which recuces $e_\mu$. The same $p = 0.036$ will be used resulting in $R_0 = 0.9$
while $R_0^1 = 1.44$. Running the simulation again it is shown that even though $R_0 < 1$ the
disease manages to survive indefinitely. But $R_0^1 = 1.44$ means the disease has a chance to never
die out within group 1 and is constantly spreading to the other groups from group 1. Figure
\ref{fig:exp_subgroups} shows the amount of infections per group. Group 1 has significantly more
infections and is keeping the disease alive, while group 2 and 3 have similar infection amounts.
This experiment shows how hard it is to estimate $R_0$
for complex networks, as the network might contain highly connected subgroups in which
the disease can live longer and spread to the rest of the network again.

\begin{figure}
    \centering
    \includegraphics[width=0.5\linewidth]{images/exp_subgroups.png}
    \caption{Infections per group with $p = 0.036$ and $R_0 = 0.9$, the disease never dies out}
    \label{fig:exp_subgroups}
\end{figure}


\section{Multiple Diseases}
Interesting dynamics can be observed if multiple diseases are introduced into the same
social network. Cases with two diseases where one or both of the diseases have a $R_0 < 1$
do not make for an interesting scenario as once one disease has died out the problem just
gets reduced to one disease. But if both (or more) diseases have a $R_0 > 1$ then they 
will compete against each for survival, assuming a person can only be infected by one diseases
at a time, for example because they stay at home until they are cured so they can not get
infected by another disease during that period. Since a disease needs to constantly infect
new nodes to stay alive the diseases can rot other diseases out by infecting all nodes themselfes
thus leaving no availabe nodes to infect for other diseases.

For diseases with a similar $R_0$ the two diseases will each have \~50\% of infections.
Now consider a case where one diseases $d_1$ has $R_0^1=2$ and the second $d_2$ has $R_0^2=5$.
When the diseases are viewed in isolation theoretically both are able to survive as both
$R_0^1 > 1, R_0^2>1$. Looking at the situation where both diseases are in the same network,
$d_2$ will infect significantly more people than $d_1$, about 2.5 times as many. This means
that $d_2$ will have $~\frac{5}{7}$ of total infections during the first wave. As $d_2$ has
way more infected nodes during the first wave that makes it even easier for it to spread than
$d_1$. At the borders where susceptible nodes have connections to nodes infected with $d_1$
and also to nodes infected with $d_2$, $d_2$ will infect $\frac{5}{7}$ of those nodes. This
will slowly reduce the amount of nodes infected with $d_1$ until none are left and $d_1$ has died
out even though its $R_0^1>1$. This means in networks with more than one disease the one 
with the highest $R_0$ value is the one most likely to survive.

\subsection{Experiment}
This theory can again be supported by conducting an experiment using the developed simulation
tool. The network will be the same used in the previous experiment, consisting of 3 groups with
more intra group connections than inter group connections. The exact structure is explained
in section \ref{sub:exp_network}.

Two diseases are used for this simulation:
\begin{itemize}
    \item (vaccinated) fatality rate: 0
    \item (re-, vaccinated) infection rate: 0.06 for Disease 1 / 0.04 for Disease 2
    \item infection period $t_I$: 5
    \item initial infections: 50
    \item cure chance: 1
    \item immunity period: 0
    \item infectiousness factor: 1
\end{itemize}

First the experiment is run with the two diseases separately to show that they never die out
on their own. The results can be seen in figure \ref{fig:exp_multiple_diseases_individual}
which shows the amount of new infections until cycle 5000 indicating that the individual 
diseases have not died out until then.

\begin{figure*}
    \centering
    \begin{subfigure}[b]{0.475\textwidth}
        \centering
        \includegraphics[width=\textwidth]{images/exp_multiple_diseases_d1.png}
        \caption[Network2]%
        {{\small Infection counts for disease 1 with 50 initial infections and $p = 0.06$}}   
    \end{subfigure}
    \hfill
    \begin{subfigure}[b]{0.475\textwidth}  
        \centering 
        \includegraphics[width=\textwidth]{images/exp_multiple_diseases_d2.png}
        \caption[]%
        {{\small Infection counts for disease 2 with 150 initial infections and $p = 0.04$}}    
    \end{subfigure}
    \caption[Experiment to show both diseases never die out on their own]
    {\small Experiment to show both diseases never die out on their own} 
    \label{fig:exp_multiple_diseases_individual}
\end{figure*}

Now both diseseases are used at the same time. After 40 cycles
$d_2$ has died out because the new infections of $d_1$ have increased so much that
not enough nodes are availabe for $d_2$ to infect. This increase in infections with $d_1$ and
decrease in infections with $d_2$ is shown in figure \ref{fig:exp_multiple_diseases}.

\begin{figure}
    \centering
    \includegraphics[width=0.5\linewidth]{images/exp_multiple_diseases_both.png}
    \caption{Infections per disease with $p = 0.06$ and $p = 0.04$, disease 2 dies out after 40 steps even though it did not die out when the two diseases were simulated in isolation}
    \label{fig:exp_multiple_diseases}
\end{figure}

This supports the theory that in networks with multiple highly infectious diseases the
one with the highest $R_0$ value is the one most likely to survive.

\section{Small-World Phenomenon}
The small-world phenomenon is also known as six degrees of separation. The theory is that
in an arbitratily large social network the length of the path from one person to any other
person is on average only 6 steps long. To support this theory Stanley Milgram conducted
an experiment in the 1960s \cite{smallWorld}. In this experiment Milgram gave a letter to 
a random source person in Nebraska (US) and told them to deliver it to a target person in
Massachusetts (US). The source person was only given basic information about the target like 
address and occupation and each person was only allowed to send the letter to someone they
knew on a first name basis with the goal to get the letter closer to its target. Any person
in this chain was given the same information. After many iteration of this experiment the 
average amount of persons it took to deliver the letter to its target was between 5 and 6 persons
leading to the name of the six degrees of separation principle.

This experiment does not always find the shortest path though as the people forward the letter
only to the person they assume to be closest to the target. Unkownst to them there might
be another person they know who is much closer to the target making for a shorter path which
is never discovered. To find the true shortest path each participant would have to forward
the letter to all their friends while keeping track of which path the letter has taken.
Then after all letters have arrived at the target the true shortest path is the one of the
letter that needed the least amount of persons to arrive.

\subsection{Experiment}
For this experiment a network with only one group is created. That group
contains 10,000 nodes with each node having 6
connections. Though this network is not totally accurate to a real situation. Usually 
a friends network of a person is tightly coupled, the friends of a person usually also know
each other and their friends know the initial person etc. Also a person mostly knows
others that live in an area close to them and only fewer persons that live farther away.
This leads to a highly connected network with only short connections and a few random longer connections.
The network used for this experiment uses only random edges which does not represent this
fact that the geolocically closer people are the more connections they have. However in the
current day the internet allows for way more connections over long distances which makes
the used network more valid in that context.

\begin{figure}
    \centering
    \includegraphics[width=0.5\linewidth]{images/sw_true_network.png}
    \caption{True structure of a network showing the friends of persons. The
    geolocically closer people are the higher the amount of connections. There
    are only few connections over longer range. (source: \cite{networks})}
    \label{fig:oscillation}
\end{figure}


To show that each node can be reached in only six steps, a disease with an infection rate
of 1 and infection duration of 1000 is used. Initially only one
node is infected. Figure \ref{fig:small_world_network} shows the network after 6 cycles.

\begin{figure}
    \centering
    \includegraphics[width=0.5\linewidth]{images/small_world_network.png}
    \caption{Network after 6 cycles, yellow nodes have a shortest connection with 6 or less steps to the starting node,
    green ones have a shotest connection with more than 6 edges to the starting node}
    \label{fig:small_world_network}
\end{figure}

After cycle 7 all nodes are infected which means they were all able to be reached from the 
starting node in a maximum of 7 steps. The exact amounts of new infections per step are:
6, 30, 150, 717, 2893, 5306 and 897. Using these values the average shortest path length
can be calculated, it is \~5.5964. This coincides with the findings of Milgram \cite{smallWorld}
whose experiments showed the average length is somewhere between 5 and 6.

To reduce the limitation of the used network in regards to the random connections another
network structure could be used. This one closer models the existance of highly coupled
local friend groups where everyone knows each other and a few random connections to people
from other friend groups who are further away. This network uses a high amount of
groups with relatively few members each. In this case 100 groups of 5-10 people are used
where each node is connected to 4 others of the same group and each group of people is
connected to 3-4 other groups with only 0-1 edges per node. This again results in a network
where on average each node has 6 connections.

The same disease is used in this network and the state of the network after cycle 6
can be seen in figure \ref{fig:small_world_groups}.

\begin{figure}
    \centering
    \includegraphics[width=0.5\linewidth]{images/small_world_groups.png}
    \caption{Network after 6 cycles, yellow nodes have a shortest connection with 6 or less steps to the starting node,
    green ones have a shotest connection with more than 6 edges to the starting node}
    \label{fig:small_world_groups}
\end{figure}

This shows that in this network it is also possible to reach all other nodes in only six steps,
even though there are only a few long distance connections and a lot of thightly coupled
small groups. The average path length in this experiment was \~5.6621




\chapter{Summary and Outlook}
\label{cha:summary}
Algorithmic Game Theory is a rapidly growing field. In this paper the basic principles such as different types of games, solution strategies, Nash equilibria and computational problems were introduced. Today there are still many research topics and developments in this area. Things like whether or not PPAD-complete problems are easier than NP-complete problems are still not definitely proven.

Other current research areas within Algorithmic Game Theory include the design of algorithmic mechanisms that are robust to strategic manipulation and on the computational complexity of these mechanism design problems, a topic also covered by Nisan et al. \autocite{agt}.

Another area of research is dynamic games, which are games that evolve over time. These games can be used to model concepts such as auctions or negotiations. Negotiations, for example, have applications in many areas, from social areas to cloud computing. The current research focuses on developing algorithms to analyze these games, as well as understanding the impact of time and strategic behavior in these games.

Machine Learning is another area of growing importance and is closely related to Algorithmic Game Theory. Current research focuses on the intersection of these two fields and the development of algorithms for learning games that can be used to analyze the behavior of machine learning algorithms. The use of Machine Learning to improve game theoretic analysis is another area of current research. Dey \autocite{MLAndAGT} shows an example use case for combining Machine Learning and Algorithmic Game Theory and Hazra and Anjaria \autocite{agtInDL} discuss applications of Algorithmic Game Theory in the field of Deep Learning.

Overall Algorithmic Game Theory includes many areas that still have interesting topics that are currently being researched. The field is constantly evolving and only growing in importance, making it a very interesting area with many topics still available for research.

% \chapter{Introduction}
% \label{cha:introduction}
% \section{Motivation}
An epidemic outbreak can have a major impact on the world. Past epidemics have shown that
if we do not know how to respond to an epidemic outbreak and are not prepared for 
such cases, a new disease can wipe out large parts of the world. The black
death caused the death of about 30\% to 60\% of all Europeans (75-200 million)
during the 1300s \cite{blackDeath}.
To better understand such scenarios, simulations play an important role, as the study of real
cases of epidemic outbreaks is difficult since there are only so many in the history of humans.
Also, in the event of an outbreak, it is important to be able to simulate the next few days/weeks
in order to accurately predict how the epidemic will evolve.

For this reason this work will discuss an approach to simulate such epidemics by modeling
networks of people and then simulating the spreading of diseases with different characteristics
in these networks.

\section{Problem definition}
A model of the social network in which the disease is spreading in is crucial to the simulation.
Depending on the transmission method of the disease, this network may be highly connected in 
the case of a disease with airborne transmission or have only few connections for diseases
that are sexually transmitted. In addition, different diseases can spread
completely different in the same social network even if they have the same transmission method
because the characteristics of the disease also play an important role in how it spreads. 
As suggested by Easley and Kleinberg \cite{networks}
the transmission of computer viruses works in a similar way can therefore also be modeled
using networks.

The networks that can be created using the app must be able to model different social
networks. The most important part for the epidemics simulation is the modeling of the amount
of contacts with other people each person has. Since each group can have a significant number
of members an efficient method for creating networks with large amounts of nodes is required that
still allows to model most social networks.

A method to visualize these networks in a clear way is needed. The visual 
representation must still be usable with large amounts of nodes (e.g. over 100,000 nodes).
To make the visualization of the network more usable, some settings must be provided
to modify the displayed network, such as hiding certain connections or nodes. It also needs
ways to represent the current state of the network in respect to the spread of the diseases.

The app also needs to allow for creation of multiple diseases with different characteristics.
To simulate various epidemic scenarios properties like the infectiousness, duration of illness
or fatality need to be editable. 

The app should be able to simulate multiple diseases at the same time. The simulation needs
to take into account which humans have contact with each other and then simulate the spreading
of the diseases according to the characteristics of each disease and group of people.

During the simulation the app will collect statistics that allow a review of key information
after the simulation, such as the number of new infections over time.

% \chapter{Terminology}
% \label{cha:terminology}
% \input{tex/terminology}

% \chapter{Defining different types of Games}
% \label{cha:gameDefinitions}
% \input{tex/gameDefinitions}

% \chapter{Solution Strategies and Nash Equilibria}
% \label{cha:solutionStrategies}
% \input{tex/solutionStrategies}

% \chapter{Complexity of finding Nash equilibria}
% \label{cha:complexityOfFindingNash}
% \input{tex/complexityOfFindingNash}

% \chapter{Summary}
% \label{cha:summary}
% Algorithmic Game Theory is a rapidly growing field. In this paper the basic principles such as different types of games, solution strategies, Nash equilibria and computational problems were introduced. Today there are still many research topics and developments in this area. Things like whether or not PPAD-complete problems are easier than NP-complete problems are still not definitely proven.

Other current research areas within Algorithmic Game Theory include the design of algorithmic mechanisms that are robust to strategic manipulation and on the computational complexity of these mechanism design problems, a topic also covered by Nisan et al. \autocite{agt}.

Another area of research is dynamic games, which are games that evolve over time. These games can be used to model concepts such as auctions or negotiations. Negotiations, for example, have applications in many areas, from social areas to cloud computing. The current research focuses on developing algorithms to analyze these games, as well as understanding the impact of time and strategic behavior in these games.

Machine Learning is another area of growing importance and is closely related to Algorithmic Game Theory. Current research focuses on the intersection of these two fields and the development of algorithms for learning games that can be used to analyze the behavior of machine learning algorithms. The use of Machine Learning to improve game theoretic analysis is another area of current research. Dey \autocite{MLAndAGT} shows an example use case for combining Machine Learning and Algorithmic Game Theory and Hazra and Anjaria \autocite{agtInDL} discuss applications of Algorithmic Game Theory in the field of Deep Learning.

Overall Algorithmic Game Theory includes many areas that still have interesting topics that are currently being researched. The field is constantly evolving and only growing in importance, making it a very interesting area with many topics still available for research.

%-----------------------------------------------------------------------
\appendix

%---
\printbibliography[heading=bibintoc]

\end{document}